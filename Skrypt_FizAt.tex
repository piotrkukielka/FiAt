\documentclass[a4paper,12pt]{article}
\usepackage{polski}
\usepackage[utf8]{inputenc}
\usepackage[T1]{fontenc}
\usepackage{mathptmx}
\usepackage{graphicx}
\usepackage{amsmath}
\usepackage{listings}
\usepackage{import}
\usepackage{float}
\usepackage{color} %red, green, blue, yellow, cyan, magenta, black, white
\usepackage[dvipsnames]{xcolor}
\usepackage{hyperref} %[hidelinks] - chowa hiperzłącza
\usepackage{pdfpages}
\usepackage[font=small]{caption}
\usepackage{titling}
\usepackage{indentfirst}
\usepackage{geometry}
\newgeometry{tmargin=1.9cm, bmargin=1.9cm, lmargin=1.9cm, rmargin=1.9cm}
\usepackage{setspace}
\usepackage{wrapfig}
\usepackage{multirow}
\usepackage{amsfonts}
\usepackage{subcaption}
\usepackage{physics}
\usepackage{bigints}
\usepackage{mathtools} %adds abs

\title{Atomowa Fizyka, czyli każdego dla skrypt} % Title

\author{Debiliusz, Kukiełak} % Author name

\date{\today} % Date for the report
\begin{document}
	\maketitle
    \textit{Ten skrypt jest nieszczęśliwym tworem wyobraźni dwóch studentów}. \\
    Zawiera tonę suchego humoru, błędów w kodzie, niekompilujących się 
    fragmentów, a co najważniejsze, jeszcze więcej błędów w tekście.\\
    Najważniejszy postulat: \textbf{wszystko brać z dystansem!!}
    
\section{Wykrzyknik 1}\label{sec:1}
	Klasyczne równania ruchu 
		$$
		\left\{
			\begin{array}{l}
				m_1 \ddot{\vec{r}}_1 = - \vec{F} \\
				m_2 \ddot{\vec{r}}_2 = \vec{F}
			\end{array}	
		\right.
		$$
	Wprowadzamy nowe zmienne
		$$
		\left\{
			\begin{array}{l}
				\vec{r} = \vec{r}_2-\vec{r}_1\\
				\vec{R} = \frac{1}{m_1 + m_2}\left(\vec{r}_1 m_1 + \vec{r}_2 m_2\right)
			\end{array}
		\right.
		$$
	przekształcamy ze względu na $\vec{r}_1,\,\vec{r}_2$
		$$
		\left\{
			\begin{array}{l}
				\vec{r}_1 = \vec{R} - \frac{m_2}{m_1+m_2}\vec{r} = \vec{R} - \frac{\mu}{m_1}\vec{r}\\
				\vec{r}_2 = \vec{R} + \frac{m_1}{m_1+m_2}\vec{r} = \vec{R} + \frac{\mu}{m_2}\vec{r}
			\end{array}
		\right.
		$$
	różniczkując dwukrotnie
		$$
		\left\{
			\begin{array}{l}
				\ddot{\vec{r}}_1 = \ddot{\vec{R}} -  \frac{\mu}{m_1}\ddot{\vec{r}}\\
				\ddot{\vec{r}}_2 = \ddot{\vec{R}}  + \frac{\mu}{m_2}\ddot{\vec{r}}
			\end{array}
		\right.
		$$
	otrzymane wektory wprowadzamy do początkowego układu
		$$
		\left\{
			\begin{array}{l}
				m_1\left( \ddot{\vec{R}} -  \frac{\mu}{m_1}\ddot{\vec{r}} \right)= - \vec{F} \\
				m_2\left(\ddot{\vec{R}}  + \frac{\mu}{m_2}\ddot{\vec{r}} \right)= \vec{F}
			\end{array}	
		\right.
		$$
	w pierwszym kroku dodajemy do siebie równania, dzięki czemu otrzymujemy
		$$
			m_1\ddot{\vec{R}} + m_2\ddot{\vec{R}} = \vec{0}
		$$
	czyli
		$$
			(m_1+m_2)\ddot{\vec{R}} = \vec{0}
		$$
	aby otrzymać drugie równanie, podzielmy równanie uprośćmy równania przez masy $m_1$ oraz $m_2$, a następnie odejmijmy je od siebie
		$$
			\ddot{\vec{R}} - \frac{\mu}{m_1}\ddot{\vec{r}} - \ddot{\vec{R}} - \frac{\mu}{m_2}\ddot{\vec{r}} = -\frac{\vec{F}}{m_1} - \frac{\vec{F}}{m_2}
		$$
	co prowadzi do
		$$
		\frac{m_1+m_2}{m_1 m_2}\mu \ddot{\vec{r}} = \frac{1}{\mu} \vec{F}
		$$
	stąd otrzymujemy
		$$
		\mu\ddot{\vec{r}} = \vec{F}
		$$
	co kończy wyprowadzenie równań ruchu.\\
	Musimy udowodnić, że wielkości
		$$
		\begin{array}{l}
			\vec{p}_{tot} = \vec{p}_1+\vec{p}_2=(m_1+m_2)\vec{v}_{CM}\\\\
			E_{tot} = \frac{m_1 v_1^2}{2}+\frac{m_2 v_2^2}{2} + V(r) = (m_1+m_2)\frac{v_{CM}^2}{2}+\frac{\mu v^2}{2} + V(r)\\\\
			\vec{L}_{tot} = \vec{r}_1\times\vec{p}_1 + \vec{r}_2\times\vec{p}_2 =  (m_1+m_2)\vec{R}\times\vec{v}_{CM}+\mu\vec{r}\times\vec{v}
		\end{array}
		$$
	gdzie $\vec{v}=\dot{\vec{r}},\quad \vec{v}_{CM} = \dot{\vec{R}}$, są
	całkami ruchu, czyli $\frac{d f}{dt}=0$. Dla momentu pędu
	$$
	\begin{array}{lll}
		\frac{d}{dt}\vec{L}_{tot} &=& \underbrace{\dot{\vec{r}}_1 \times\vec{p}_1}_{=0} + \vec{r}_1\times\dot{\vec{p}}_1 + \underbrace{\dot{\vec{r}}_2 \times\vec{p}_2}_{=0} + \vec{r}_2\times\dot{\vec{p}}_2 = \\\\
		&=&\bigg\vert \dot{\vec{p}}_1 = -\vec{F}, \quad \dot{\vec{p}}_2 = \vec{F} \bigg\vert = \vec{r}_1\times(-\vec{F})+\vec{r}_2\times\vec{F} = (\vec{r}_2-\vec{r}_1)\times \vec{F} = 0
	\end{array}
	$$
	Dla energi całkowitej, zapisanej za pomocą pędów 
	$$
	\begin{array}{lll}
		\frac{d}{dt}E_{tot} & = & \frac{\vec{p}_1\circ\ddot{\vec{p}}_1}{m_1} + \frac{\vec{p}_2\circ\ddot{\vec{p}}_2}{m_2} + \dot{V}(r) =
		\bigg\vert \dot{\vec{p}}_1 = -\vec{F}, \quad \dot{\vec{p}}_2 = \vec{F},\quad dV = \grad{V(r)}\circ d\vec{r} \bigg\vert =\\\\
		&=& \vec{F}\circ(\vec{v}_2-\vec{v}_1) + \grad{V(r)}\circ\dot{\vec{r}} = \vec{F}\circ \vec{v} + \vec{v}\circ(-\vec{F}) = 0
	\end{array}
	$$
	Oraz najprostsze dla pędu całkowitego
	$$
	\dot{\vec{p}}_{tot} = \dot{\vec{p}}_1 + \dot{\vec{p}}_2 = -\vec{F}+\vec{F} = 0
	$$

\section{Wykrzyknik 2}
Wektor Rungego-Lentza
\[
  \vec{A} = \vec{p} \times \vec{L} + \mu D \hat{r}
\]
gdzie $D = kq_1q_2$ oraz $\vec{p} = \mu \dot{\vec{r}}$.
Musimy pokazać, że $\dot{\vec{A}} = 0$. Pamiętając z paragrafu~\ref{sec:1}, że
$\dot{\vec{L}} = 0$ bierzemy pochodną iloczynu:
\[
  \dot{\vec{A}} = \dot{\vec{p}} \times \vec{L} + \mu D \dot{\hat{r}} = \mu
  \ddot{\vec{r}} \times \vec{L} + \mu D \dot{\hat{r}} = \vec{F} \times \vec{L} +
  \mu D \dot{\hat{r}}
\]
Wiemy, że $V = \frac{D}{r}$ więc $\vec{F} = -grad\vec{V} =
\frac{-D}{r^2}\hat{r}$. Kontynuując:
\[
  \dot{\vec{A}} = \frac{D}{r^2}\hat{r} \times \left( \mu \vec{r} \times
  \dot{\vec{r}} \right) + \mu D \dot{\hat{r}} = \mu D\hat{r} \times \left(
  \hat{r} \times
  \dot{\hat{r}} \right) + \mu D \dot{\hat{r}} = \mu D \left[\underbrace{\left( \hat{r} \circ
    \dot{\hat{r}} \right)}_{=0} \hat{r} - \underbrace{\left( \hat{r} \circ \hat{r}
  \right)}_{=1}
\dot{\hat{r}} + \dot{\hat{r}} \right] = \vec{0}
\]

\subsection{Komentarze}
Na wykładzie wektor Rungego-Lentza jest od razu pomnożony przez $\mu$.

Użyte wzory:
\[
		\vec{a}\times(\vec{b}\times\vec{c}) = \vec{b} (\vec{a}\circ\vec{c}) -
    \vec{c}(\vec{a}\circ\vec{b})
\]
oraz należy pamiętać, że $\vec{r} = r\hat{r}$, stąd $\dot{\vec{r}} =
\dot{r}\hat{r} + r\dot{\hat{r}}$.

\section{Wykrzyknik 3}
	We wstępie nie mamy podanego wektora $\vec{S}_p$, zatem chyba trzeba go wyprowadzić, więc
	$$
	\vec{S}_p = \frac{1}{\mu_0}\vec{E}_p \times \vec{B}_p = \frac{1}{\mu_0 c}\vec{E}_p \cross(\hat{R}\cross\vec{E}_p)
	$$
	ciągnąc równość dalej
	$$
	= \frac{1}{\mu_0 c}\left(\hat{R}E_p^2 - \vec{E}_p\underbrace{(\hat{R}\circ\vec{E}_p)}_{=0}\right) = \bigg\vert E_p^2 = \frac{k^2 q^2}{c^4 R^2} \left[\hat{R}(\hat{R}\times \vec{a})\right]^2 = \frac{k^2 q^2}{c^4 R^2} |\hat{R}\times\vec{a}|^2\bigg\vert = \frac{1}{\mu_0 c}\frac{k^2q^2}{c^4 R^2}\left[a^2-(\hat{R}\circ\vec{a})^2\right]\hat{R}
	$$
	co kończy wyprowadzenie $\vec{S}_p$. Wtedy obliczamy $P$
	$$
	\begin{array}{lll}
		P &=& \bigointsss\limits_\Sigma \vec{S}_p(\vec{r},t )d\vec{\sigma} = \frac{A}{R^2}\oint\limits_\Sigma\left[a^2 - (\hat{R}\circ\vec{a})^2\right]\hat{R}\hat{R}d\sigma = \bigg\vert d\sigma = R^2 \underbrace{\sin\theta d\theta d\varphi}_{d\Omega} \bigg\vert =\\\\\\
		&=& \frac{A}{R^2}R^2 a^2 \bigintss\limits_{4\pi} d\Omega -\frac{A}{R^2}R^2 \underbrace{\int\limits_{4\pi}(\hat{R}\circ\vec{a})^2d\Omega}_I
	\end{array}
	$$
	Zajmujemy się całką $I$, przyjmując, że $\vec{a} || OZ\implies \vec{a}\circ\hat{R} = a\cos\theta$, dostajemy
	$$
		I=\int\limits_\Omega\, a^2\cos^2\theta\sin\theta d\theta d\varphi = \bigg\vert z = \cos\theta\implies dz = -\sin\theta d\theta\bigg\vert = 2\pi a^2 \int\limits_{-1}^1 z^2 dz = \frac{4\pi a^2}{3}
	$$
	biorąc wszystko razem
	$$
	P = 4\pi A a^2 - \frac{4\pi a^2 A}{3} = \frac{8\pi}{3}a^2 A = \frac{8\pi}{3}a^2 \frac{k^2 q^2}{c^5 \mu_0} = \frac{8\pi}{3}a^2 \frac{1}{4\pi\varepsilon_0\mu_0}\frac{kq^2}{c^5} =
	= \frac{2}{3}a^2 \frac{kq^2}{c^3} = \bigg\vert \alpha = \frac{ke^2}{\hbar c}\bigg\vert = \frac{2}{3}\alpha\frac{(q/e)^2\hbar}{c^2}a^2
	$$
	\subsection{Komentarze}
		Użyte wzory:
		$$
		\vec{a}\times(\vec{b}\times\vec{c}) = \vec{b} (\vec{a}\circ\vec{c}) - \vec{c}(\vec{a}\circ\vec{b})
		$$
		oraz 
		$$
		|\vec{a} \times\vec{b}|^2 = a^2b^2 - (\vec{a}\circ\vec{b})^2
		$$

\section{Wykrzyknik 4}
Postulaty modelu Bohra:
\begin{enumerate}
  \item istnieją stacjonarne orbity - znajdujący się na nich elektron nie
    wypromieniowuje energii elektromagnetycznej.
  \item elektron emituje foton przy zmianie orbity $h\nu= \abs*{E_f - E_i}$
  \item moment pędu przyjmuje dyskretne wartości $ L = n\hbar$
\end{enumerate}
Z trzeciego postulatu otrzymujemy $E_n = \frac{-1}{2n^2}E_0$, gdzie $E_0 =
\frac{k^2 e^4 m_e}{\hbar}$
i jest to praktycznie jedyny poprawny wynik w tym modelu. 

Całej reszty nie robiliśmy. (?)


\section{Wykrzyknik 5}
		Musimy dokonać zamiany zmiennych w $\hat{H}$ na zmienne środka masy. Obliczając, jak poprzednio
		$$
		\left\{
			\begin{array}{l}
				\vec{r} = \vec{r}_2-\vec{r}_1\\
				\vec{R} = \frac{1}{m_1 + m_2}\left(\vec{r}_1 m_1 + \vec{r}_2 m_2\right)
			\end{array}
		\right.
	$$

		wtedy prawdziwa jest następująca reguła łańcuchowa 
		$$
		\left\{
			\begin{array}{l}
				\nabla_1 = -\nabla_r + \frac{\mu}{m_2}\nabla_R \\
				\nabla_2 = \nabla_r + \frac{\mu}{m_1}\nabla_R 
			\end{array}
		\right.
		$$
		wprowadzając do Hamiltonianu
		$$
		\begin{array}{lll}
			\hat{H}_{tot} &=& -\frac{\hbar^2}{2m_2}\nabla_2^2 - \frac{\hbar^2}{2m_1}\nabla_1^2 + V(r) = -\frac{\hbar^2}{2m_2}\left[\nabla_r^2+\left(\frac{\mu}{m_1}\right)^2\nabla_R^2
			+\frac{2\mu}{m_1}\nabla_R\nabla_r \right] -\frac{\hbar^2}{2m_1}\left[\nabla_r^2+\left(\frac{\mu}{m_2}\right)^2\nabla_R^2
			-\frac{2\mu}{m_2}\nabla_R\nabla_r \right] + \\
			&+& V(r) = -\frac{\hbar^2}{2\mu}\nabla_r^2-\frac{\hbar^2}{2M}\nabla_R^2 + V(r) = \frac{\hat{\vec{p}}_{CM}^2}{2M} + \underbrace{\frac{\hat{\vec{p}}^2}{2\mu} +  V(r)}_{\hat{H}}
		\end{array}
		$$
		co kończy pierwszą część.\\
		Poprawka liniowa do masy zredukowanej
		$$
			\mu = \frac{m_em_p}{m_e+m_p} \cdot \frac{\frac{1}{m_p}}{\frac{1}{m_p}} = \frac{m_e}{1+\frac{m_e}{m_p}} \approx m_e \left(1-\frac{m_e}{m_p}\right)
		$$
		wiemy, że $m_e \sim 10^{-31},\quad m_p \sim 10^{-27}$ [kg], zatem $m_e/m_p ~\sim 10^{-4}$.

\section{Wykrzyknik 6}
Wychodząc z równania Schrodingera:
\[
  \hat{H} \phi(\vec{r})  = E \phi (\vec{r})
\]
\[
  \frac{-\hbar^2}{2m}\nabla^2 \phi (\vec{r}) - \frac{Zke^2}{r}\phi (\vec{r}) = E
\phi (\vec{r})
\]
Mnożymy obustronnie przez $\frac{-2m}{\hbar^2}$, rozpisujemy laplasjan oraz
wprowadzamy nowe oznaczenia by uniknąć przepisywania stałych $\beta^2 =
\frac{-2m}{\hbar^2}E$, $U(r) = \frac{2m}{\hbar^2}\frac{Zke^2}{r}$ oraz
$\hat{O}_r = \frac{\partial}{\partial r}r^2 \frac{\partial}{\partial r}$.
Następnie przenosimy całość na jedną stronę.

\[
  \left[ \frac{1}{r^2} \hat{O}_r - \frac{1}{\hbar^2 r^2}\hat{\vec{L}}^2 + U(r) -
\beta^2 \right] \phi (\vec{r}) = 0
\]
Mnożymy przez $r^2$
\[
  \left[ \hat{O}_r - \frac{1}{\hbar^2}\hat{\vec{L}}^2 + r^2U(r) -
r^2 \beta^2 \right] \phi (\vec{r}) = 0
\]
Korzystamy z możliwości rozdzielenia funkcji falowej $\phi (\vec{r}) =
R(\vec{r}) Y(\theta, \phi) = R Y$
\[
  Y\left( \hat{O}_r + r^2U(r) -
  r^2 \beta^2 \right)R - \frac{R}{\hbar^2}\hat{\vec{L}}^2Y = 0
\]
Mnożymy $\frac{1}{RY}$
\[
  \underbracket{\frac{1}{R}\left( \hat{O}_r + r^2U(r) -
    r^2 \beta^2 \right)R}_{=-\lambda} - \underbracket{\frac{1}{\hbar^2
Y}\hat{\vec{L}}^2Y}_{=\lambda} = 0
\]
Stąd 
\[
  \frac{1}{\hbar^2}\hat{\vec{L}} = \lambda Y \qquad \Rightarrow \qquad \lambda = l(l+1)
\]
a dalej
\[
  \left(\frac{1}{r^2}\hat{O}_r + U(r) - \beta^2 + l(l+1)\frac{1}{r^2} \right)R = 0
\]
po rozpisaniu otrzymujemy
\[
  R'' + \frac{2}{r}R' + \frac{2m}{\hbar^2}\left(\frac{Zke^2}{r} -
  \frac{l(l+1)}{r} \frac{h^2}{2mr} + E \right) R = 0
\]

\section{Wykrzyknik 8}
$E_n = \frac{-Z^2}{2n^2}E_0$ zależy tylko od jednej liczby kwantowej $n=1, 2,
...$. Zatem wartości własne $\hat{H}$ są zdegenerowane.
Można policzyć stopień tej degeneracji:
\[
  \sum^{n-1}_{l=0} (2l+1) 
\]
Co można policzyć ze wzoru na sumę ciągu arytmetycznego (pamiętając, że
startujemy na $l=0$ więc ilość wyrazów to $n$.
\[
  \sum^{n-1}_{l=0} (2l+1) = \frac{1+2(n-1) + 1}{2}n = n^2
\]

\section{Wykrzyknik 10}
nie byo?
\end{document}
