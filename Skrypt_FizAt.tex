% !TeX spellcheck = de_DE
\documentclass[a4paper,12pt]{article}
\usepackage{polski}
\usepackage[utf8]{inputenc}
\usepackage[T1]{fontenc} %% TMP
\usepackage{palatino}
\usepackage{graphicx}
\usepackage{amsmath}
\usepackage{listings}
\usepackage{import}
\usepackage{float}
\usepackage{color} %red, green, blue, yellow, cyan, magenta, black, white
\usepackage{hyperref} %[hidelinks] - chowa hiperzłącza
\usepackage{pdfpages}
\usepackage[font=small]{caption}
\usepackage{titling}
\usepackage{indentfirst}
\usepackage{geometry}
\newgeometry{tmargin=1.9cm, bmargin=1.9cm, lmargin=1.9cm, rmargin=1.9cm}
\usepackage{setspace}
\usepackage{wrapfig}
\usepackage{multirow}
\usepackage{amsfonts}
\usepackage{subcaption}
\usepackage{physics}
\usepackage{bigints}
\usepackage{mathtools} %adds abs
\usepackage{enumitem} %adds letters to items in enumerate

\usepackage{xcolor}
\newcommand\myworries[1]{\textcolor{red}{#1}} %package and macro for comments

\title{Atomowa Fizyka, czyli każdego dla skrypt} % Title

\author{Debiliusz, Kukiełak, Maciek Kalka} % Author name

\date{\today} % Date for the report
\begin{document}
	\maketitle
  \textit{Ten skrypt jest nieszczęśliwym tworem wyobraźni dwóch (i pół) studentów}. \\
    Zawiera tonę suchego humoru, błędów w kodzie, niekompilujących się 
    fragmentów, a co najważniejsze, jeszcze więcej błędów w tekście.\\
    Najważniejszy postulat: \textbf{wszystko brać z dystansem!!}
    
\section{Wykrzyknik 1}\label{sec:1}
	Klasyczne równania ruchu 
		$$
		\left\{
			\begin{array}{l}
				m_1 \ddot{\vec{r}}_1 = - \vec{F} \\
				m_2 \ddot{\vec{r}}_2 = \vec{F}
			\end{array}	
		\right.
		$$
	Wprowadzamy nowe zmienne
		$$
		\left\{
			\begin{array}{l}
				\vec{r} = \vec{r}_2-\vec{r}_1\\
				\vec{R} = \frac{1}{m_1 + m_2}\left(\vec{r}_1 m_1 + \vec{r}_2 m_2\right)
			\end{array}
		\right.
		$$
	przekształcamy ze względu na $\vec{r}_1,\,\vec{r}_2$
		$$
		\left\{
			\begin{array}{l}
				\vec{r}_1 = \vec{R} - \frac{m_2}{m_1+m_2}\vec{r} = \vec{R} - \frac{\mu}{m_1}\vec{r}\\
				\vec{r}_2 = \vec{R} + \frac{m_1}{m_1+m_2}\vec{r} = \vec{R} + \frac{\mu}{m_2}\vec{r}
			\end{array}
		\right.
		$$
  gdzie 
    \begin{equation*}
      \mu = \frac{m_1 m_2}{m_1 + m_2}
    \end{equation*}
	różniczkując dwukrotnie
		$$
		\left\{
			\begin{array}{l}
				\ddot{\vec{r}}_1 = \ddot{\vec{R}} -  \frac{\mu}{m_1}\ddot{\vec{r}}\\
				\ddot{\vec{r}}_2 = \ddot{\vec{R}}  + \frac{\mu}{m_2}\ddot{\vec{r}}
			\end{array}
		\right.
		$$
	otrzymane wektory wprowadzamy do początkowego układu
		$$
		\left\{
			\begin{array}{l}
				m_1\left( \ddot{\vec{R}} -  \frac{\mu}{m_1}\ddot{\vec{r}} \right)= - \vec{F} \\
				m_2\left(\ddot{\vec{R}}  + \frac{\mu}{m_2}\ddot{\vec{r}} \right)= \vec{F}
			\end{array}	
		\right.
		$$
	w pierwszym kroku dodajemy do siebie równania, dzięki czemu otrzymujemy
		$$
			m_1\ddot{\vec{R}} + m_2\ddot{\vec{R}} = \vec{0}
		$$
	czyli
		$$
			(m_1+m_2)\ddot{\vec{R}} = \vec{0}
		$$
	aby otrzymać drugie równanie, podzielmy równania z otrzymanego układu równań odpowiednio przez masy $m_1$ oraz $m_2$, a następnie odejmijmy je od siebie
		$$
			\ddot{\vec{R}} - \frac{\mu}{m_1}\ddot{\vec{r}} - \ddot{\vec{R}} - \frac{\mu}{m_2}\ddot{\vec{r}} = -\frac{\vec{F}}{m_1} - \frac{\vec{F}}{m_2}
		$$
	co prowadzi do
		$$
		\frac{m_1+m_2}{m_1 m_2}\mu \ddot{\vec{r}} = \frac{1}{\mu} \vec{F}
		$$
	stąd otrzymujemy
		$$
		\mu\ddot{\vec{r}} = \vec{F}
		$$
	co kończy wyprowadzenie równań ruchu.\\
	Musimy udowodnić, że wielkości
		$$
		\begin{array}{l}
			\vec{p}_{tot} = \vec{p}_1+\vec{p}_2=(m_1+m_2)\vec{v}_{CM}\\\\
			E_{tot} = \frac{m_1 v_1^2}{2}+\frac{m_2 v_2^2}{2} + V(r) = (m_1+m_2)\frac{v_{CM}^2}{2}+\frac{\mu v^2}{2} + V(r)\\\\
			\vec{L}_{tot} = \vec{r}_1\times\vec{p}_1 + \vec{r}_2\times\vec{p}_2 =  (m_1+m_2)\vec{R}\times\vec{v}_{CM}+\mu\vec{r}\times\vec{v}
		\end{array}
		$$
	gdzie $\vec{v}=\dot{\vec{r}},\quad \vec{v}_{CM} = \dot{\vec{R}}$, są
	całkami ruchu, czyli $\frac{d f}{dt}=0$. Dla momentu pędu
	$$
	\begin{array}{lll}
		\frac{d}{dt}\vec{L}_{tot} &=& \underbrace{\dot{\vec{r}}_1 \times\vec{p}_1}_{=0} + \vec{r}_1\times\dot{\vec{p}}_1 + \underbrace{\dot{\vec{r}}_2 \times\vec{p}_2}_{=0} + \vec{r}_2\times\dot{\vec{p}}_2 = \\\\
		&=&\bigg\vert \dot{\vec{p}}_1 = -\vec{F}, \quad \dot{\vec{p}}_2 = \vec{F} \bigg\vert = \vec{r}_1\times(-\vec{F})+\vec{r}_2\times\vec{F} = (\vec{r}_2-\vec{r}_1)\times \vec{F} = 0
	\end{array}
	$$
	Dla energi całkowitej, zapisanej za pomocą pędów 
	$$
	\begin{array}{lll}
		\frac{d}{dt}E_{tot} & = & \frac{\vec{p}_1\circ\ddot{\vec{p}}_1}{m_1} + \frac{\vec{p}_2\circ\ddot{\vec{p}}_2}{m_2} + \dot{V}(r) =
		\bigg\vert \dot{\vec{p}}_1 = -\vec{F}, \quad \dot{\vec{p}}_2 = \vec{F},\quad dV = \grad{V(r)}\circ d\vec{r} \bigg\vert =\\\\
		&=& \vec{F}\circ(\vec{v}_2-\vec{v}_1) + \grad{V(r)}\circ\dot{\vec{r}} = \vec{F}\circ \vec{v} + \vec{v}\circ(-\vec{F}) = 0
	\end{array}
	$$
	Oraz najprostsze dla pędu całkowitego
	$$
	\dot{\vec{p}}_{tot} = \dot{\vec{p}}_1 + \dot{\vec{p}}_2 = -\vec{F}+\vec{F} = 0
	$$
DODATEK: WYPROWADZENIE L
(trzykropek - to samo ale dla ''2'')
\begin{equation*}
  \begin{split}
    L_{tot} &= \vec{r}_1 \times \vec{p}_1 + \vec{r}_2 \times \vec{p}_2 = \left(
    \vec{R} - \frac{\mu}{m_1}\vec{r} \right) \times m_1 \left(
  \dot{\vec{R}} - \frac{\mu}{m_1} \hat{\vec{r}} \right) + \cdots = \\
  &= m_1 \left[ \vec{r} \times \dot{\vec{r}} - \frac{\mu}{m_1} \vec{r} \times
    \dot{\vec{r}} - \frac{\mu}{m_1} \vec{r} \times \dot{\vec{r}}  -
  \frac{\mu}{m_1} \frac{\mu^2}{m_1^2}\vec{r} \times \dot{\vec{r}} \right]
  + \\
  &+ m_2 \left[ \vec{r} \times \dot{\vec{r}} - \frac{\mu}{m_2} \vec{r} \times
    \dot{\vec{r}} - \frac{\mu}{m_2} \vec{r} \times \dot{\vec{r}}  -
  \frac{\mu}{m_2} \frac{\mu^2}{m_2^2}\vec{r} \times \dot{\vec{r}} \right]
  =\\ &= \left(m_1 + m_2\right) \vec{R} \times \dot{\vec{R}} + \mu^2 \vec{r}
  \times \dot{\vec{r}} \left( \frac{m_1}{m_1^2} + \frac{m_2}{m_2^2} \right) = \left(
    m_1 + m_2 \right) \vec{R} \times \dot{\vec{R}} + \mu \vec{r} \times
    \dot{\vec{r}}
  \end{split}
\end{equation*}

\section{Wykrzyknik 2}
Wektor Rungego-Lentza
\[
  \vec{A} = \vec{p} \times \vec{L} + \mu D \hat{r}
\]
gdzie $D = kq_1q_2$ oraz $\vec{p} = \mu \dot{\vec{r}}$.
Musimy pokazać, że $\dot{\vec{A}} = 0$. Pamiętając z paragrafu~\ref{sec:1}, że
$\dot{\vec{L}} = 0$ bierzemy pochodną iloczynu:
\[
  \dot{\vec{A}} = \dot{\vec{p}} \times \vec{L} + \mu D \dot{\hat{r}} = \mu
  \ddot{\vec{r}} \times \vec{L} + \mu D \dot{\hat{r}} = \vec{F} \times \vec{L} +
  \mu D \dot{\hat{r}}
\]
Wiemy, że $V = \frac{D}{r}$ więc $\vec{F} = -grad\vec{V} =
\frac{-D}{r^2}\hat{r}$. Kontynuując:
\[
  \dot{\vec{A}} = -\frac{D}{r^2}\hat{r} \times \left( \mu \dot{\vec{r}} \times
  \vec{r} \right) + \mu D \dot{\hat{r}} = \mu D\hat{r} \times \left(
  \hat{r} \times
  \dot{\hat{r}} \right) + \mu D \dot{\hat{r}} = \mu D \left[\underbrace{\left( \hat{r} \circ
    \dot{\hat{r}} \right)}_{=0} \hat{r} - \underbrace{\left( \hat{r} \circ \hat{r}
  \right)}_{=1}
\dot{\hat{r}} + \dot{\hat{r}} \right] = \vec{0}
\]

\myworries{TODO: \@b) znajdź jego kierunek i zwrot (dużo rysunów, pierwszy wykład i pierwsze ćwiczenia)}

\subsection{Komentarze}
Na wykładzie wektor Rungego-Lentza jest od razu pomnożony przez $\mu$.

Użyte wzory:
\[
		\vec{a}\times(\vec{b}\times\vec{c}) = \vec{b} (\vec{a}\circ\vec{c}) -
    \vec{c}(\vec{a}\circ\vec{b})
\]
oraz należy pamiętać, że $\vec{r} = r\hat{r}$, stąd $\dot{\vec{r}} =
\dot{r}\hat{r} + r\dot{\hat{r}}$.

\section{Wykrzyknik 3}
  DODATEK:
  Wyprowadzenie ogolne $\vec{S}$. Z danych zadania możemy otrzymać
  \begin{equation*}
    \vec{S} = \frac{1}{\mu} \left[ \underbrace{\vec{E}_c \times
        \vec{B}_c}_{~\frac{1}{r^4}} + \underbrace{\vec{E}_c \times
        \vec{B}_p}_{~\frac{1}{r^3}} + \underbrace{\vec{E}_p \times
        \vec{B}_c}_{~\frac{1}{r^3}} + \underbrace{\vec{E}_p \times
    \vec{B}_p}_{~\frac{1}{r^2}} \right] = \vec{S}_p + \cdots
  \end{equation*}
  Koniec dodatku\\
	We wstępie nie mamy podanego wektora $\vec{S}_p$, zatem chyba trzeba go wyprowadzić, więc
	$$
	\vec{S}_p = \frac{1}{\mu_0}\vec{E}_p \times \vec{B}_p = \frac{1}{\mu_0 c}\vec{E}_p \cross(\hat{R}\cross\vec{E}_p)
	$$
	ciągnąc równość dalej
	$$
	= \frac{1}{\mu_0 c}\left(\hat{R}E_p^2 -
  \vec{E}_p\underbrace{(\hat{R}\circ\vec{E}_p)}_{=0}\right) = \bigg\vert E_p^2 =
  \frac{k^2 q^2}{c^4 R^2} \left[\hat{R}(\hat{R}\times \vec{a})\right]^2 =
  \frac{k^2 q^2}{c^4 R^2} |\hat{R}\times\vec{a}|^2\bigg\vert =
  \underbrace{\frac{1}{\mu_0 c}\frac{k^2q^2}{c^4
  R^2}}_{~\frac{const}{R^2}}\left[a^2-(\hat{R}\circ\vec{a})^2\right]\hat{R}
	$$
	co kończy wyprowadzenie $\vec{S}_p$. Wtedy obliczamy $P$
	$$
	\begin{array}{lll}
		P &=& \bigointsss\limits_\Sigma \vec{S}_p(\vec{r},t )d\vec{\sigma} = \frac{A}{R^2}\oint\limits_\Sigma\left[a^2 - (\hat{R}\circ\vec{a})^2\right]\hat{R}\hat{R}d\sigma = \bigg\vert d\sigma = R^2 \underbrace{\sin\theta d\theta d\varphi}_{d\Omega} \bigg\vert =\\\\\\
		&=& \frac{A}{R^2}R^2 a^2 \bigintss\limits_{4\pi} d\Omega -\frac{A}{R^2}R^2 \underbrace{\int\limits_{4\pi}(\hat{R}\circ\vec{a})^2d\Omega}_I
	\end{array}
	$$
	Zajmujemy się całką $I$, przyjmując, że $\vec{a} || OZ\implies \vec{a}\circ\hat{R} = a\cos\theta$, dostajemy
	$$
		I=\int\limits_\Omega\, a^2\cos^2\theta\sin\theta d\theta d\varphi = \bigg\vert z = \cos\theta\implies dz = -\sin\theta d\theta\bigg\vert = 2\pi a^2 \int\limits_{-1}^1 z^2 dz = \frac{4\pi a^2}{3}
	$$
	biorąc wszystko razem
	$$
	P = 4\pi A a^2 - \frac{4\pi a^2 A}{3} = \frac{8\pi}{3}a^2 A = \frac{8\pi}{3}a^2 \frac{k^2 q^2}{c^5 \mu_0} = \frac{8\pi}{3}a^2 \frac{1}{4\pi\varepsilon_0\mu_0}\frac{kq^2}{c^5} =
	= \frac{2}{3}a^2 \frac{kq^2}{c^3} = \bigg\vert \alpha = \frac{ke^2}{\hbar c}\bigg\vert = \frac{2}{3}\alpha\frac{(q/e)^2\hbar}{c^2}a^2
	$$
	\myworries{TODO: podać wartości liczbowe}
	\subsection{Komentarze}
		Użyte wzory:
		$$
		\vec{a}\times(\vec{b}\times\vec{c}) = \vec{b} (\vec{a}\circ\vec{c}) - \vec{c}(\vec{a}\circ\vec{b})
		$$
		oraz 
		$$
		|\vec{a} \times\vec{b}|^2 = a^2b^2 - (\vec{a}\circ\vec{b})^2
		$$

\section{Wykrzyknik 4}
		Musimy dokonać zamiany zmiennych w $\hat{H}$ na zmienne środka masy. Obliczając, jak poprzednio
		$$
		\left\{
			\begin{array}{l}
				\vec{r} = \vec{r}_2-\vec{r}_1\\
				\vec{R} = \frac{1}{m_1 + m_2}\left(\vec{r}_1 m_1 + \vec{r}_2 m_2\right)
			\end{array}
		\right.
	$$

		wtedy prawdziwa jest następująca reguła łańcuchowa 
		$$
		\left\{
			\begin{array}{l}
				\nabla_1 = -\nabla_r + \frac{\mu}{m_2}\nabla_R \\
				\nabla_2 = \nabla_r + \frac{\mu}{m_1}\nabla_R 
			\end{array}
		\right.
		$$
		wprowadzając do Hamiltonianu
		$$
		\begin{array}{lll}
			\hat{H}_{tot} &=& -\frac{\hbar^2}{2m_2}\nabla_2^2 - \frac{\hbar^2}{2m_1}\nabla_1^2 + V(r) = -\frac{\hbar^2}{2m_2}\left[\nabla_r^2+\left(\frac{\mu}{m_1}\right)^2\nabla_R^2
			+\frac{2\mu}{m_1}\nabla_R\nabla_r \right] -\frac{\hbar^2}{2m_1}\left[\nabla_r^2+\left(\frac{\mu}{m_2}\right)^2\nabla_R^2
			-\frac{2\mu}{m_2}\nabla_R\nabla_r \right] + \\
			&+& V(r) = -\frac{\hbar^2}{2\mu}\nabla_r^2-\frac{\hbar^2}{2M}\nabla_R^2 + V(r) = \frac{\hat{\vec{p}}_{CM}^2}{2M} + \underbrace{\frac{\hat{\vec{p}}^2}{2\mu} +  V(r)}_{\hat{H}}
		\end{array}
		$$
		co kończy pierwszą część.\\
		Poprawka liniowa do masy zredukowanej
		$$
			\mu = \frac{m_em_p}{m_e+m_p} \cdot \frac{\frac{1}{m_p}}{\frac{1}{m_p}} = \frac{m_e}{1+\frac{m_e}{m_p}} \approx m_e \left(1-\frac{m_e}{m_p}\right)
		$$
		wiemy, że $m_e \sim 10^{-31},\quad m_p \sim 10^{-27}$ [kg], zatem $m_e/m_p ~\sim 10^{-4}$.

\section{Wykrzyknik 5}
Wychodząc z równania Schrodingera:
\[
  \hat{H} \phi(\vec{r})  = E \phi (\vec{r})
\]
\[
  \frac{-\hbar^2}{2m}\nabla^2 \phi (\vec{r}) - \frac{Zke^2}{r}\phi (\vec{r}) = E
\phi (\vec{r})
\]
Mnożymy obustronnie przez $\frac{-2m}{\hbar^2}$, rozpisujemy laplasjan oraz
wprowadzamy nowe oznaczenia by uniknąć przepisywania stałych $\beta^2 =
\frac{-2m}{\hbar^2}E$, $U(r) = \frac{2m}{\hbar^2}\frac{Zke^2}{r}$ oraz
$\hat{O}_r = \frac{\partial}{\partial r}r^2 \frac{\partial}{\partial r}$.
Następnie przenosimy całość na jedną stronę.

\[
  \left[ \frac{1}{r^2} \hat{O}_r - \frac{1}{\hbar^2 r^2}\hat{\vec{L}}^2 + U(r) -
\beta^2 \right] \phi (\vec{r}) = 0
\]
Mnożymy przez $r^2$
\[
  \left[ \hat{O}_r - \frac{1}{\hbar^2}\hat{\vec{L}}^2 + r^2U(r) -
r^2 \beta^2 \right] \phi (\vec{r}) = 0
\]
Korzystamy z możliwości rozdzielenia funkcji falowej $\phi (\vec{r}) =
R(\vec{r}) Y(\theta, \phi) = R Y$
\[
  Y\left( \hat{O}_r + r^2U(r) -
  r^2 \beta^2 \right)R - \frac{R}{\hbar^2}\hat{\vec{L}}^2Y = 0
\]
Mnożymy $\frac{1}{RY}$
\[
  \underbracket{\frac{1}{R}\left( \hat{O}_r + r^2U(r) -
    r^2 \beta^2 \right)R}_{=-\lambda} - \underbracket{\frac{1}{\hbar^2
Y}\hat{\vec{L}}^2Y}_{=\lambda} = 0
\]
Stąd 
\[
  \frac{1}{\hbar^2}\hat{\vec{L}} Y = \lambda Y \qquad \Rightarrow \qquad \lambda = l(l+1)
\]
a dalej
\[
  \left(\frac{1}{r^2}\hat{O}_r + U(r) - \beta^2 + l(l+1)\frac{1}{r^2} \right)R = 0
\]
po rozpisaniu otrzymujemy
\[
  R'' + \frac{2}{r}R' + \frac{2m}{\hbar^2}\left(\frac{Zke^2}{r} -
  \frac{l(l+1)}{r} \frac{h^2}{2mr} + E \right) R = 0
\]
\section{Wykrzyknik 6}
Trochę inne wyprowadzenie tego co już robiliśmy, wprowadzamy zmienną $\rho = \frac{r}{a_0},\quad \varepsilon = \frac{E}{E_0}$ oraz nową funkcję
$f(\rho) = \rho R(a_0 \rho)$, wtedy
$$
\begin{array}{l}
	\frac{d}{dr}R(r) = \frac{1}{a_0}\left(\frac{f'}{\rho} - \frac{f}{\rho^2}\right)\\\\
	\frac{d^2}{dr^2}R(r) = \frac{1}{a_0^2}\left(\frac{f''}{\rho}-2\frac{f'}{\rho^2}+2\frac{f}{\rho^3}\right)
\end{array}
$$
wprowadzamy to do równania (pomijam argumenty!)
$$
\begin{array}{l}
	R'' +\frac{2}{r}R' + \frac{2m}{\hbar^2}\left(E+\frac{ke^2}{r}-\frac{\hbar^2l(l+1)}{2mr^2}\right)R = 0 \to \\\\
	\to \frac{1}{a_0^2}\left(\frac{f''}{\rho}-2\frac{f'}{\rho^2}+2\frac{f}{\rho^3}\right) + \frac{2}{\rho a_0^2}\left(\frac{f'}{\rho} - \frac{f}{\rho^2}\right)
	+\frac{2m}{\hbar^2}\left(E+\frac{ke^2}{a_0\rho}-\frac{\hbar^2l(l+1)}{2ma_0^2\rho^2}\right)\frac{f}{\rho} = 0
\end{array}
$$
Czynniki zawierające $f', f$ z policzenia $R'',R'$ zredukują się pozostawiając jedynie $f''$, wtedy, mnożąc przez $\rho a_0^2$ otrzymujemy
$$
	f''  +\frac{2ma_0^2}{\hbar^2}\left(E+\frac{ke^2}{a_0\rho}-\frac{\hbar^2l(l+1)}{2ma_0^2\rho^2}\right)f = 0
$$
włączamy $2$ do nawiasu, jednocześnie wyciągając $E_0$ dzięki czemu otrzymujemy
$$
	f''  +\frac{ma_0^2}{\hbar^2}\frac{mk^2e^4}{\hbar^2}\left(2\varepsilon+\frac{ke^2}{a_0\rho}\frac{\hbar^2}{mk^2e^4}-\frac{\hbar^2l(l+1)}{ma_0^2\rho^2}\frac{\hbar^2}{mk^2e^4}\right)f = 0
$$
korzystamy z definicji $a_0$ dostając wynik 
$$
	f'' + \left(2\varepsilon+\frac{2}{\rho}- \frac{l(l+1)}{\rho^2}\right)f = 0
$$
\myworries{TODO:Zmienić "prawie to co robiliśmy" na "to co robiliśmy" czyli zamieniać zmienne tak jak on chce. Generalnie cały do przepisania. Dodać nowy podpunkt}


\section{Wykrzyknik 7}
$E_n = \frac{-Z^2}{2n^2}E_0$ zależy tylko od jednej liczby kwantowej $n=1, 2,
...$. Zatem wartości własne $\hat{H}$ są zdegenerowane.
Można policzyć stopień tej degeneracji:
\[
  \sum^{n-1}_{l=0} (2l+1) 
\]
Co można policzyć ze wzoru na sumę ciągu arytmetycznego (pamiętając, że
startujemy na $l=0$ więc ilość wyrazów to $n$.
\[
  \sum^{n-1}_{l=0} (2l+1) = \frac{1+2(n-1) + 1}{2}n = n^2
\]
Pełna funkcja falowa atomu H-podobnego
$$
\psi(\vec{r})_{n,l,m_l} = R_{n,l}(r) Y_{l,m_l}(\hat{r})
$$
przy czym liczby kwantowe zmieniają się następująco
$$
\begin{gathered}
n = 1,2,\cdots\\
l = 0,1,\cdots,n-1\\
m_l = -l,-l+1,\cdots,0,\dots,l-1,l
\end{gathered}
$$
Korzystając z faktu unormowania
$$
\begin{gathered}
1=\bra{\psi_{n,l,m_l}}\ket{\psi_{n,l,m_l}} = \int dr\, r^2 R_{n,l}R_{n,l} \int d\Omega Y^*_{l,m_l}Y_{l,m_l} =  \int\, r^2 dr R_{n,l}R_{n,l} \delta_{l,l}\delta_{m_l,m_l} = \int dr \, r^2 R_{n,l}^2
\end{gathered}
$$
\section{Wykrzyknik 8}
No to własność harmonik, jeżeli $\hat{r} = (\theta,\varphi)$ to $-\hat{r} = (\pi-\theta,\pi+\varphi)$, zatem
$$
	Y_{lm}(-\hat{r}) = Y_{lm}(\pi-\theta,\pi+\varphi) = (-1)^m C_{lm} P^m_l[\cos(\theta-\pi)]e^{im(\pi+\varphi)} = C_{lm} P^m_l[-\cos(\theta)]e^{im\varphi}
$$
teraz korzystamy z postaci Rodrigueza dla wielomianów Legendre'a dostając
$$
	(-1)^l (-1)^m C_{lm} P^m_l[\cos(\theta)]e^{im\varphi} = (-1)^l Y_{lm}(\hat{r})
$$

\subsection{Komentarz}
	Własności dla wielomanów Legendre
	$$
		P_l^m (-x) = (-1)^{l+m} P_l^m(x)
	$$
\section{Wykrzyknik 9}
Ponieważ 
\[
  \int_0^\infty dr r^2 R_{nl}^2(r) = 1
\]
więc możemy przyjąć, że funkcja $p_{nl} = r^2R_{nl}^2(r)$ jest gęstością
prawdopodobieństwa znalezienia elektronu w odległości r od jądra. (bez dowodu
na wykładzie i nie zadane jako wykrzyknik, więc dowód jest improwizowany)
Dowód:
\[
  p_{nl} = r^2 \int_0^{2\pi}\phi\int_0^\pi d\theta\sin\theta
  \abs{\phi(\vec{r})}^2 = \bigg\vert \phi_{nlm} (\vec{r}) =
  R_{nl}(r)Y_{lm}(\hat{r}) \bigg\vert =
  r^2R_{nl}^2(r)\int_0^{2\pi}\phi\int_0^\pi d\theta\sin\theta Y^*_{lm} Y_{l'm'}
  = r^2 R_{nl}^2(r)
  \
\]
Podpunkty
\begin{enumerate}[label=(\alph*)]
  \item Korzystając z danych z zadania
    \[
      R_{10} = 2e^{-2}
    \]
    \[
      p_{10} = r^2(2e^{-r})^2 = 4r^2 e^{-2r}
    \]
  \item
    \[
      \left< r \right>_{10} = 4 \int_0^\infty r^3 e^{-2r} dr = \bigg \vert t =
      2r \bigg \vert = \frac{1}{4} \int_0^\infty t^3 e^{-t} dt = \frac{3}{2}
    \]
    Wychodząc z jednostek atomowych $a_0 \approx 0.5 \dot{A}$ $\Rightarrow$
    $\left< r \right>_{10} \approx \frac{3}{4} \dot{A}$
\end{enumerate}
\section{Wykrzyknik 10}
$$
	\phi_{210} = R_{21}(r) Y_{10}(\hat{r})
$$
Normujemy
$$
	\braket{\phi_{210}}{\phi_{210}} = \int\limits_{\mathbb{R}^3} R_{21}^2 Y_{10} Y_{10}^\dagger d^3r
$$
Funkcja $\phi$ jest iloczynem dwóch funkcji, o rozdzielonych zmiennych, zatem mamy do policzenia 2 całki
$$
	= \int\limits_0^{\infty}R^2 r^2 dr \cdot \int\limits_{\Omega} Y_{10} Y_{10}^\dagger d\Omega
$$
najpierw całka radialna
$$
	\int\limits_0^{\infty}R^2 r^2 dr  = C^2\int\limits_0^{\infty}r^4 e^{-r} = 8C^2
$$
teraz całka kątowa, jednak harmoniki tworzą układ ortonormalny, zatem
$$
	\int\limits_{\Omega} Y_{10} Y_{10}^\dagger d\Omega = 1
$$
wtedy obliczamy stałą $C$
$$
	8C^2 = 1 \implies C= \frac{1}{2\sqrt{6}}
$$
teraz liczymy średnią odległość
$$
	\bra{\phi_{210}}r\ket{\phi_{210}} = \int\limits_0^{\infty}R^2 r^3 dr \cdot \int\limits_{\Omega} Y_{10} Y_{10}^\dagger d\Omega
$$
pomijamy część kątową
$$
	\frac{1}{24}\int\limits_0^{\infty}r^5 e^{-r} = \frac{5!}{24} = 5 \to r_{21} = 5a_0
$$

\section{Wykrzyknik 11}
Ogólnie:
\[
  \vec{r} \equiv  r \sqrt{\frac{4\pi}{3}}\sum_{m=-1}^{1} Y_{1m}
  \vec{\epsilon}_m^*
\]
Więc musimy to udowodnić. Korzystamy z podanych informacji oraz z postaci
harmonik sferycznych:
\[
  Y_{10} = \sqrt{\frac{3}{4\pi}} \cos(\theta)
\]
\[
  Y_{1, \pm 1} = \pm \sqrt{\frac{3}{8\pi}} \sin(\theta) e^{\pm i \phi}
\]
Możemy zapisać, rozpisując sumę:

\begin{equation*}
    \vec{r}  = \sqrt{\frac{4\pi}{3}} r \left[
    \sqrt{\frac{3}{8\pi}}\sin(\theta)e^{-i\phi}\frac{1}{\sqrt{2}} \left(
    \hat{e}_1 - i\hat{e}_2 \right) + \sqrt{\frac{3}{4\pi}}\cos(\theta)\hat{e}_3 +
    \sqrt{\frac{3}{8\pi}}\sin(\theta)e^{i\phi} \frac{1}{\sqrt{2}}\left(\hat{e}_1 +
  i\hat{e}_2 \right) \right]
\end{equation*}
Wyciągając przed nawias i skracając ze sobą współczynniki  oraz wymnażając i
porządkując równanie:
\begin{equation*}
  \vec{r} = r \left[ \hat{e}_1 \frac{1}{2} \sin(\theta) \underbrace{\left( e^{-i\phi} +
    e^{i\phi} \right)}_{2\cos(\phi)} + \hat{e}_2 i \frac{1}{2}\sin(\theta)
    \underbrace{\left( e^{i\phi} - e^{-i\phi} \right)}_{2i\sin(\phi)} +
  \hat{e}_3 \cos(\theta)\right]
\end{equation*}
Stąd wymnażając
\begin{equation*}
  \vec{r} = r\left( \sin(\theta) \cos(\theta) \hat{e}_1 + \sin(\theta)
  \sin(\phi) \hat{e}_2 + \cos(\theta) \hat{e}_3 \right) = \vec{r}
\end{equation*}
Co kończy dowód. Przykład użycia: dowód, że $\sum_{m_k} \abs{\vec{d}_{jk}}^2$ nie
zależy od $m_j$.

Dowód ostatniej tożsamości pozostawiam jako ćwiczenie dla czytelnika. Proponuję
rozwiązanie poprzez rozpisanie wszystkich przypadków bądź znalezienie ogólnego
wzoru na $\epsilon$.
\section{Wykrzyknik 12}
Reguła wyboru dla zmiany $l$
$$
	\bra{\varphi_{n_1l_1m_1}}\left[\hat{\vec{L}}^2,[\hat{\vec{L}}^2,\vec{r}]\right]\ket{\varphi_{n_2l_2m_2}} = 2\hbar^2\bra{\varphi_{n_1l_1m_1}}\left\{\hat{\vec{L}}^2,\vec{r}\right\} \ket{\varphi_{n_2l_2m_2}}
$$
Obliczamy lewą stronę ($L$)
$$
\begin{array}{lll}
	L &=& \bra{\varphi_{n_1l_1m_1}} \hat{\vec{L}}^2 \hat{\vec{L}}^2\vec{r}- 2\hat{\vec{L}}^2\vec{r}\hat{\vec{L}}^2+\vec{r}\hat{\vec{L}}^2\hat{\vec{L}}^2\ket{\varphi_{n_2l_2m_2}} =\\\\
	&=& \hbar^4\left[l_1^2(l_1+1)^2-2l_1(l_1+1)l_2(l_2+1)+l_2^2(l_2+1)^2 \right]\vec{r}_{12} = \hbar^4\left[l_1(l_1+1)-l_2(l_2+1)\right]^2\vec{r}_{12}
\end{array}
$$
następnie prawa strona ($P$)
$$
\begin{array}{lll}
	P &=& 2\hbar^2\bra{\varphi_{n_1l_1m_1}}\hat{\vec{L}}^2\vec{r}+\vec{r}\hat{\vec{L}}^2\ket{\varphi_{n_2l_2m_2}} = 2\hbar^4\left[l_1(l_1+1)+l_2(l_2+1)\right]\vec{r}_{12}
\end{array}
$$
przyrównujemy je do siebie $L=P$, pomijamy wektor $\vec{r}_{12}, \hbar^4$, w odpowiednim momencie wprowadzimy zmienne $u = l_1+l_2,\quad v = l_1-l_2\implies l_1=\frac{1}{2}(u+v),\quad l_2 = \frac{1}{2}(u-v)$
$$
	L = (...) = (l_1^2+l_1-l_2^2-l_2)^2 = [(l_1-l_2)(l_1+l_2)+l_1-l_2]^2 = (l_1-l_2)^2 (l_1+l_2+1)^2 = v^2 (u+1)^2
$$
oraz
$$
	P = 2[l_1(l_1+1)+l_2(l_2+1)] = 2l_1^2+2l_2^2+2(l_1+l_2) = \frac{1}{2}(u+v)^2+\frac{1}{2}(u-v)^2+2u = u^2+v^2+2u
$$
przepisujemy w postaci $L-P=0$
$$
\begin{array}{lll}
	L-P &=& v^2(u+1)^2-u^2-v^2-2u = v^2(u+1-1)(u+1+1) - u(u+2) = v^2u(u+2)-u(u+2) =\\\\ &=& u(u+2)(v-1)(v+1) = (l_1+l_2)(l_1+l_2+2)[(l_1-l_2)^2-1] = 0
\end{array}
$$
co daje reguły wyboru.

\section{Wykrzyknik 13}
Hamiltonian atomu wodoru(bez spinu) w polu elektro-magnetycznym ma następującą postać:
$$H_{wod} = \frac{1}{2m}(\hat{\vec{P}}-Q\vec{A})^{2} + Q\varphi$$
Gdzie $\vec{A}$ - potencjał wektorowy, $\varphi$ - potencjał skalarny. \\
Dla elektronu w polu magnetycznym i w polu jądra(sytuacja w atomie wodoru w polu B):
\begin{align*}
\varphi(\vec{r}) = \frac{ke^2}{r} && \vec{B} = \curl{\vec{A}}
\end{align*}

Dla stałego w czasie i przestrzeni pola B zachodzi
\begin{align*}
\vec{A} = \frac{1}{2} \vec{B} \times \vec{r} && \div{\vec{A}} = 0
\end{align*}

Wtedy, pracując w cechowaniu Coulomba, gdzie A komutuje z p
\begin{align*}
(\hat{\vec{P}}+e\vec{A})^{2} &= \vec{p}^2 +2e\vec{A}\vec{p} e^2\vec{A}^2 \\
							 &= \vec{p}^2 +2e\vec{A}\vec{p} \\
							 &= \vec{p}^2 + e(\vec{B} \times \vec{r})\vec{p} \\
							 &= \vec{p}^2 + e(\vec{p} \times \vec{r})\vec{B} \\
							 &= \vec{p}^2 + e\vec{L}\vec{B}
\end{align*}

Ostatecznie hamiltonian atomu wodoru, w stałym polu B, bez spinu wynosi:
$$\hat{H} = \frac{1}{2m}(\vec{p}^2 + e\hbar \frac{\vec{L}}{\hbar} \vec{B}) - \frac{ke^2}{r}$$

Z rachunku zaburzeń:
$$H= H_{at} + \lambda\hat{V}$$
gdzie $H_{at}$ to hamiltonian bez pola, a zaburzenie(operator oddziaływania z polem) jest postaci
\begin{align*}
\lambda \hat{V} &= \frac{e}{\hbar} \vec{B} \frac{\vec{L}}{\hbar} \\
				&= \mu_0  \vec{B} \frac{\vec{L}}{\hbar}
\end{align*} 

Pracując przy ustalonym n w podprzestrzeni degeneracji $\varphi_{nml}(\vec{r})$ - bazie niezaburzonych wektorów własnych hamiltoninanu $H_{at}$ zdegenerowanej $n^2$ krotnie - możemy obliczyć element macierzowy operatora oddziaływania z polem(zaburzenia). \\
W ramach stałego w przestrzeni pola B wybieramy jego kierunek. Niech $\vec{B} || OZ \rightarrow \vec{B}(\vec{r}) \equiv B_z$. Dodatkowo dla uproszczenia funkcje falowe w braketach zapisuję tylko przy pomocy indeksów.

\begin{align*}
V^0_{nm} &= \matrixel{nl'm'}{\lambda \hat{V}}{nlm} \\
		 &= \mu_0 B_z \matrixel{nl'm'}{\frac{\hat{L}}{\hbar}}{nlm} \\
		 &= \mu_0 B_z m \braket{nl'm'}{nlm} \\
		 &= \mu_0 B_z m \delta_{l'l}\delta_{m'm}
\end{align*}

Zatem operator jest diagonalny, a co za tym idzie baza wektorów własnych hamiltonianu bez pola jest bazą dopasowaną do zaburzenia. Ostatecznie
$$E_{nm} =  -\frac{E_0}{n^2} + \mu_0 B_z m$$
A zatem wartość poprawki energetycznej przy stałym w czasie i przestrzeni polu B wynosi
$$\Delta_E = \mu_0 B_z m $$

Dla stanu $n=3$, $l=0, 1, 2$ dostajemy $m_0 = 0$ - brak zmiany energii, $m_1 = 0, \pm1$ - zmiana egergii o $\Delta_1 = \pm \mu_0 B_z m$ i $m_2 = 0, \pm 1, \pm 2$. Rysunki w zeszycie(mozna wstawic skan)

\section{Wykrzyknik 14}
	Przyjmujemy jednostki atomowe oraz następującą numerację 
		$$
		\begin{array}{lll}
			\varphi_1 &=& \varphi_{200} = \frac{1}{\sqrt{2}}e^{-r/2}\left(1-r/2\right)Y_{00}\\
			\varphi_2 &=& \varphi_{210} = \frac{1}{2\sqrt{6}}re^{-r/2}Y_{10}\\
			\varphi_3 &=& \varphi_{211} = \frac{1}{2\sqrt{6}}re^{-r/2}Y_{11}\\
			\varphi_4 &=& \varphi_{21-1}= \frac{1}{2\sqrt{6}}re^{-r/2}Y_{1-1}
		\end{array}
		$$
	Musimy policzyć elementy macierzowe 
		$$
		a_{ij} = \bra{\varphi_i}E_z z\ket{\varphi_j}
		$$
	Lecimy- najpierw elementy zerowe, przy czym  $z = r\cos(\theta) = r\sqrt{\frac{4\pi}{3}}Y_{10}$
		\begin{enumerate}
			\item  $a_{11} = \bra{\varphi_1}E_z z\ket{\varphi_1} = 0$ przez: całka z $\cos(\theta)$ po przedziale $[0,\pi]$ 
			\item $a_{13} = \bra{\varphi_1}E_z z\ket{\varphi_3} = 0$ przez: całka z $e^{i\varphi}$ po przedziale $[0,2\pi]$
			\item $a_{14} = \bra{\varphi_1}E_z z\ket{\varphi_4} =0 $ przez:  całka z $e^{-i\varphi}$ po przedziale $[0,2\pi]$
			\item $a_{22} = \bra{\varphi_2}E_z z\ket{\varphi_2} = 0$ przez: całka z $\cos^3(\theta)\sin(\theta)$ po przedziale $[0,\pi]$ 
			\item $a_{23} = \bra{\varphi_2}E_z z\ket{\varphi_3} = 0$ przez: całka z $e^{i\varphi}$ po przedziale $[0,2\pi]$
			\item $a_{24} = \bra{\varphi_2}E_z z\ket{\varphi_4} = 0$ przez: całka z $e^{-i\varphi}$ po przedziale $[0,2\pi]$
			\item $a_{33} = \bra{\varphi_3}E_z z\ket{\varphi_3} = 0$ przez: całka z $\sin^3(\theta)\cos(\theta)$ po przedziale $[0,\pi]$
			\item $a_{34} = \bra{\varphi_3}E_z z\ket{\varphi_4} = 0$ przez: całka z $e^{2i\varphi}$ po przedziale $[0,2\pi]$
			\item $a_{44} = \bra{\varphi_4}E_z z\ket{\varphi_4} = 0$ przez: całka z $\sin^3(\theta)\cos(\theta)$ po przedziale $[0,\pi]$
		\end{enumerate}
	natomiast dla elementów niezerowych, pamiętając, że $z = r\cos(\theta) = r\sqrt{\frac{4\pi}{3}}Y_{10}$
		$$
		\begin{array}{lll}
			a_{12} &=& E_z \bra{\varphi_1}z\ket{\varphi_2} = \int\limits_{\mathbb{R}^3}d^3r\varphi_1^*r\cos(\theta)\varphi_2 = \\\\
				   &=& E_z \int\limits_0^\infty dr r^3 \frac{1}{\sqrt{2}}e^{-r/2}\left(1-r/2\right)
				   \frac{1}{2\sqrt{6}}re^{-r/2}\int\limits_{\Omega}\sqrt{\frac{4\pi}{3}}Y_{10}Y_00^*Y_{10} = \\\\
				   &=& E_z \frac{1}{4\sqrt{3}}\int\limits_0^\infty dr r^4\left(1-r/2\right)e^{-r} \int\limits_{\Omega}\sqrt{\frac{1}{3}}Y_{10}Y_{10} =\\\\
				   &=& E_z \frac{1}{8\sqrt{3}}\int\limits_0^\infty dr 2r^4e^{-r}-r^5e^{-r} \frac{1}{\sqrt{3}} = \\\\
				   &=& E_z \frac{1}{24}(2*4!-5!) = -3 E_z
		\end{array}
		$$
		
\subsection{Komentarze}
	Całka postaci 
	$$
		\Gamma(x) = \int\limits_0^\infty e^{-t}t^{x-1}dt 
	$$
	dla $x=n\in\mathbb{N}$ upraszcza się do
	$$
		\Gamma(n) = \int\limits_0^\infty e^{-t}t^{n-1}dt =(n-1)!
	$$

\section{Wykrzyknik 15}
Szukamy wartości własnych czyli odejmujemy lambdy na diagonali i przy użyciu
rozwinięcia Laplace'a względem elementu 44 a następnie 33 otrzymujemy
\begin{equation*}
  \lambda^2 \left( \lambda^2 - \Delta^2 \right) = 0
\end{equation*}
stąd $\lambda_{1,2} = 0$, $\lambda_3 = \Delta$, $\lambda_4 = -\Delta$.
Znaleźliśmy wartości własne, więc wyliczamy wektory własne (z prawej dopisujemy
wektor-kolumienkę (x,y,z,t), kolejno przypadki dla każdej lamdy przyrównujemy do
kolumny zer). Wymnażając i normalizując otrzymujemy:

\begin{equation*}
  \phi_{\lambda=\Delta} = \frac{1}{\sqrt{2}}\begin{pmatrix} 1 \\ 1 \\ 0 \\ 0 \end{pmatrix}\quad
  \phi_{\lambda = -\Delta} = \frac{1}{\sqrt{2}}\begin{pmatrix} 1 \\ -1 \\ 0 \\ 0 \end{pmatrix}\quad
  \phi_{\lambda = 0} = \left\{\begin{pmatrix}1 \\ 1 \\ 0 \\
    0\end{pmatrix},\begin{pmatrix}1 \\ 1 \\ 0 \\ 0\end{pmatrix} \right\}
\end{equation*}
A więc $\phi_{\lambda_1=0} = \phi_3$, $\phi_{\lambda_2=0} = \phi_4$,
$\phi_{\lambda=\Delta} = \frac{1}{\sqrt{2}}\left( \phi_1 + \phi_2 \right)$,
$\phi_{\lambda=-\Delta} = \frac{1}{\sqrt{2}}\left( \phi_1 - \phi_2 \right)$

\myworries{TODO: Dodać: najdź poprawki pierwszego rzędu do energii E2 atomu wodoru}
\section{Wykrzyknik 16}
	Transformacja cechowania pola elektromagnetycznego
	$$
	\begin{array}{lll}
		\vec{A}' &=& \vec{A}-\grad{\chi} \\
		\phi' &=& \phi+\dot{\chi}
	\end{array}
	$$
	Wtedy pole elektryczne
	$$
		\vec{E}' = -\dot{\vec{A}}'-\grad{\phi}' = -\dot{\vec{A}}+\dot{\grad{\chi}} - \grad{(\phi+\dot{\chi})} = -\dot{\vec{A}}+\dot{\grad{\chi}} - \grad{\phi} - \grad{\dot{\chi}} = -\dot{\vec{A}} - \grad{\phi} = \vec{E}
	$$
	jest niezmiennicze oraz pole magnetyczne
	$$
		\vec{B}' = \nabla\times\vec{A}' = \nabla\times(\vec{A}-\grad{\chi}) = \nabla\times\vec{A} = \vec{B}
	$$
	Jeżeli ustalimy $\vec{B}=\vec{const}$, wówczas potencjał wektorowy $\vec{A} = \frac{1}{2} \nabla \times \vec{B}$, przyjmijmy, że $\vec{B} = (0,0,B)$, wtedy
	$$
		\vec{A} = \frac{1}{2}\left|
			\begin{array}{ccc}
				i & j & k \\
				0 & 0 & B \\
				x & y & z 
			\end{array}\right| = \frac{1}{2}(-yB,xB,0)
	$$
	natomiast skoro $\vec{B} = \nabla\times\vec{A}$:
	$$
		B = \partial_x A_y - \partial_y A_x =\frac{B}{2}+\frac{B}{2} = B
	$$
	Ponadto, spełnione jest cechowanie Culomba
	$$
		\nabla\vec{A} =\partial_x A_x + \partial_y A_y + \partial_z A_z = 0
	$$

\section{Wykrzyknik 17}
Oznaczamy 
\begin{equation*}
  \hat{D} (\vec{A}) \Psi' = \hat{\vec{P}} - Q \vec{A}
\end{equation*}
oraz
\begin{equation*}
  \gamma = e^{\frac{i}{\hbar}q\chi}
\end{equation*}
wtedy równanie Schrodingera (w cechowaniu Coulomba):
\begin{equation*}
  \left[ \frac{\hat{D}^2(\vec{A}')}{2m} + q\phi \right] \Psi' = i \hbar
  \dot{\Psi}'
\end{equation*}
Chcemy udowodnić, że
\begin{equation*}
  \hat{H}(\vec{A}', \phi') \Psi' = i\hbar \Psi' \quad \Leftrightarrow \quad
  H(\vec{A}, \phi) \Psi = i\hbar \Psi
\end{equation*}
Chcąc wyliczyć $\hat{\vec{D}}(\vec{A}')$ 
\begin{equation*}
  \hat{\vec{D}} (\vec{A} - \nabla \chi) \gamma \Psi = \left( \vec{P} - q\vec{A}
  + q\nabla \chi \right) \gamma \Psi = -i \hbar \nabla
  e^{\frac{i}{\hbar}q\chi} \Psi(\vec{r}, t) - q \vec{A} \gamma\Psi + q\nabla\chi
  \gamma \Psi
\end{equation*}
rozpisując pochodną iloczynu (należy pamiętać, że $\chi$ jest zależne od r)
oraz zwijając eksponenty do $\gamma$ otrzymujemy:
\begin{equation*}
  \hat{\vec{D}} (\vec{A} - \nabla \chi) \gamma \Psi = -\gamma \nabla \chi \Psi
  \hat{\vec{D}} (\vec{A} - \nabla \chi) \gamma \Psi = -\gamma q \nabla \chi \Psi
  - i\hbar \gamma \nabla \Psi - q\vec{A} \gamma \Psi + q\nabla \chi \gamma \Psi
  = \gamma \hat{\vec{P}} \Psi - \gamma q \vec{A} \Psi = \gamma \hat{\vec{D}}
  (\vec{A}) \Psi
\end{equation*}
Stąd 
\begin{equation*}
  \hat{D}(\vec{A}) = \gamma\hat{D} (\vec{A})\gamma^{-1}
\end{equation*}
więc
\begin{equation*}
  \hat{D}^2 (\vec{A}) = \gamma\hat{D}(\vec{A})\gamma^{-1} \gamma
  \hat{D}(\vec{A}) \gamma^{-1} = \gamma \hat{D}^2(\vec{A}) \gamma^{-1}
\end{equation*}
Dodatkowo wyliczamy pochodną:
\begin{equation*}
  \dot{\Psi} = -\frac{i}{\hbar}q \dot{\chi} \gamma\Psi + \gamma\dot{\Psi}
\end{equation*}
Wracając do równania Schrodingera
\begin{equation*}
  \left[ \frac{\gamma \hat{D}^2(\vec{A}) \gamma^{-1}
}{2m} + q\phi \right] \Psi' = i \hbar -\frac{i}{\hbar}q \dot{\chi} \gamma\Psi +
\gamma\dot{\Psi}
\end{equation*}
Wymnażając i skracając (używamy danych z tematu zadania)
\begin{equation*}
  \frac{\gamma\hat{D}(\vec{A})}{2m}\Psi + q\phi \gamma \Psi = i\hbar \gamma
  \dot{\Psi}
\end{equation*}
Mnożąc z lewej strony przez $\gamma^{-1}$
\begin{equation*}
  \left[ \frac{\hat{D}(\vec{A})}{2m} + q\phi \right]  \Psi = i\hbar 
  \dot{\Psi}
\end{equation*}
Co kończy dowód.

\section{Wykrzyknik 18}
	Rozwiązujemy podpunktami
	\begin{enumerate}
		\item Wiemy, że $\vec{A}(\vec{r},t) = \vec{A}_0 \sin(\omega t - \vec{k}\cdot\vec{r}+\varphi)$, szukamy zależnośc na parametry tak aby
		równanie $\square\vec{A}=0$ było spełnione, pamiętając $\square = \nabla^2 - \frac{1}{c^2}\partial_t^2$.\\
		Nie liczyliśmy tego na zajęciach, a ogólnie zapowiada się sporo liczenia (laplasjan z funkcji wektorowej), zatem ograniczam się do przypadku jednowymiarowego
		$$
			\partial_x^2 A = k^2 A_0 \sin(\omega t - kx +\varphi)
		$$
		oraz
		$$
			\partial_t^2 A = \omega^2 A_0\sin(\omega t - kx+\varphi)
		$$
		zatem 
		$$
			 k^2 A_0 \sin(\omega t - kx +\varphi) - \frac{\omega^2}{c^2}A_0\sin(\omega t - kx+\varphi) = 0 
		$$
		co jest spełnione wtedy i tylko wtedy, gdy $k = \frac{\omega}{c}$
		\item Warunek na $\vec{A}_0$ znajdujemy z cechowania Culomba
		$$
			\nabla \vec{A} = 0 \Leftrightarrow -k_xA_{0_x} \cos(\omega t - \vec{k}\cdot\vec{r}+\varphi) - k_yA_{0_y}\cos(\omega t - \vec{k}\cdot\vec{r}+\varphi)
			-k_z A_{0_z} \cos(\omega t - \vec{k}\cdot\vec{r}+\varphi) = 0 
		$$
		czyli 
		$$
			k_xA_{0_x} + k_yA_{0_y} + k_zA_{0_z} = \vec{A}_0 \circ \vec{k} = 0  \Leftrightarrow \vec{A}_0 \perp \vec{k}
		$$
		\item Wektor Poyntinga jest to powierzchniowa gęstość strumienia mocy fali elektromagnetycznej $[\vec{S}] = \frac{W}{m^2}$
		\item Przy obliczaniu wektorów $\vec{E},\vec{B},\vec{S}$ przyjmuję fazę $\varphi =0$, wtedy
		$$
			\vec{E} = -\dot{\vec{A}} = -\vec{A}_0\omega\cos(\omega t -\vec{k}\vec{r})
		$$
		oraz dla $\vec{B}$
		$$
		\begin{array}{l}
			\vec{B} = \nabla\times\vec{A} = \epsilon_{ijk}\hat{e}_i\partial_j A_{0_k}\sin(\omega t-\vec{k}\vec{r}) = -\epsilon_{ijk}\hat{e}_i A_{0_k}k_j\cos(\omega t-\vec{k}\vec{r}) =\\
			=\vec{A}_0\times\vec{k}\cos(\omega t-\vec{k}\vec{r})
		\end{array}
		$$
		wtedy $\vec{S}$
		$$
		\begin{array}{l}
			\vec{S}=\mu_0^{-1}\vec{E}\times\vec{B} = -\mu_0^{-1}\omega\cos^2(\omega t-\vec{k}\vec{r})\,\,\vec{A}_0\times(\vec{A}_0\times\vec{k}) =\\
			= -\mu_0^{-1}\omega\cos^2(\omega t-\vec{k}\vec{r})\,\, \left[\vec{A}_0(\vec{A}_0\circ\vec{k})-\vec{k}A_0^2\right]
		\end{array}
		$$
	\end{enumerate}
\subsection{Komentarze}
	Powyższy wykrzyknik był zrobiony częściowo w ramach wykładu, ale dla funkcji $\cos$ a nie $\sin$.\\
	Wzór na podwójny iloczyn wektorowy
	$$
	\vec{a}\times(\vec{b}\times\vec{c}) = \vec{b}(\vec{a}\circ\vec{c}) - \vec{c}(\vec{a}\circ\vec{b})
	$$

\section{Wykrzyknik 19}
Rozbijając całkę na dwie
\begin{equation*}
  \bar{\rho} = \frac{1}{T} \int_0^T \rho dt = I_1 + I_2
\end{equation*}
\begin{equation*}
  I_1 = \frac{1}{2T}\epsilon_0 \int_0^T dt E_0^2 \sin^2 (\omega t -
  \underbrace{\vec{q} \circ \vec{r}}_{=\phi}) = \frac{\epsilon_0}{2T} E_0^2
  \int_0^T \sin^2 (\omega t - \phi) dt
\end{equation*}
Podstawiając $x = \omega t - \phi$
\begin{equation*}
  I_1 = \frac{\epsilon_0 E_0^2}{2T} \int_{-\phi}^{2\pi - \phi} \sin^2xdx =
  \frac{\epsilon_0 E_0^2}{8\pi} \int_{-\phi}^{2\pi-\phi} dx =
  \frac{\epsilon_0 E_0^2}{4}
\end{equation*}
Druga całka, analogicznie:
\begin{equation*}
  I_2 = \frac{1}{2T\mu_0} \int^T_0 \sin^2 (\omega t - \vec{q} \circ \vec{r}) B_0^2 =
  \frac{B_0^2}{4 \mu_0}
\end{equation*}
Wiedząc, że
\begin{equation*}
  E_0^2 = \omega^2 A_0^2
\end{equation*}
oraz ($\vec{q} \perp A_0$ ponieważ cechowanie Coulomba)
\begin{equation*}
  B_0^2 = \left( \vec{q} \times \vec{A_0} \right)^2 = q^2 A_0^2 = \bigg \vert
  \omega = qc \bigg \vert = \frac{\omega^2}{c^2}A_0^2
\end{equation*}
Łączymy całki i podstawiamy powyższe równości
\begin{equation*}
  \bar{\rho} = \frac{1}{4}\omega^2 A_0^2 \left(\epsilon_0 + \frac{1}{\mu_0 c^2}
    = \frac{\epsilon_0}{2} \omega^2 A_0^2\right) 
\end{equation*}

\section{Wykrzyknik 20}
Żarówka, dane: $P=100$W, $l = 1$[cm], $r=1$[mm], $\nu=10^{15}$[Hz]
\begin{enumerate}
	\item $\overline{\rho}_\varepsilon  = \frac{1}{T}\int\limits_0^T\frac{1}{2}(\varepsilon_0E^2+\mu_0^{-1}B^2)dt = I_1+I_2$
	.Obliczamy kolejno
	$$
	\begin{gathered}
	I_1 = \frac{1}{2T}\int\limits_0^T\varepsilon_0E^2 dt = \frac{\varepsilon_0}{2T}\int\limits_0^TA_0^2\omega^2\sin^2(\omega t -\vec{k}\vec{r})dt = \\\\
	= \frac{E_0^2}{2T}\int\limits_0^T\sin^2(\omega t -\vec{k}\vec{r})dt = \big\vert \omega t - \varphi = u\big\vert = \\\\
	= \frac{E_0^2}{2T\omega}\int\limits_{-\varphi}^{\omega T-\varphi}\sin^2u du = \frac{\varepsilon_0 E_0^2}{4} \\\\\\
	I_2 = \frac{1}{2T\mu_0}\int\limits_0^T B^2 dt = \frac{B_0^2}{2T\mu_0}\int\limits_0^T\sin^2(\omega t -\varphi)dt = \frac{B_0^2}{4\mu_0}
	\end{gathered}
	$$
	wtedy 
	$$
	\overline{\rho}_\varepsilon = \frac{\varepsilon_0 E_0^2}{4}+ \frac{B_0^2}{4\mu_0} = \frac{1}{4}\omega^2A_0^2(\varepsilon_0+\frac{1}{\mu_0c^2})=\frac{\varepsilon_0\omega^2A_0^2}{2}
	$$
	\item $\frac{P}{2\pi r l c} = \overline{\rho}_\varepsilon = \frac{\varepsilon_0\omega^2A_0^2}{2}$
	Zatem dla naszej żarówki 
	$$
	A_0 = \frac{1}{\omega}\sqrt{\frac{P}{\pi r l c \varepsilon_0}} \sim \big\vert \hbar \omega \sim 10eV\big\vert \sim 10^{-12}
	$$
\end{enumerate}






\section{Wykrzyknik 21}
\begin{equation*}
  \left[ \hat{H}_{at}, \vec{r} \right] \phi =
  \underbrace{\hat{H}_{at}\vec{r}\phi}_{L} -
  \underbrace{\vec{r}\hat{H}_{at}\phi}_{P}
\end{equation*}
\begin{equation*}
  L = \frac{-\hbar^2 \nabla^2 }{2m} (\vec{r} \phi) - \frac{ke^2}{r}(\vec{r}
  \phi) = \frac{-\hbar}{2m }\nabla \left( \nabla \vec{r} \phi + \vec{r} \nabla
    \phi\right) - \frac{ke^2}{r}(\vec{r} \phi) = \frac{-\hbar ^2 }{2m} \left(
    \underbrace{\nabla^2 \vec{r} \phi}_{0} + 2\underbrace{\nabla \vec{r}}_{1}
  \nabla \phi  + \vec{r} \nabla^2 \phi \right) - \frac{ke^2}{r} (\vec{r} \phi)
\end{equation*}
\begin{equation*}
  P = \vec{r} \hat{H}_{at} \phi = -\vec{r} \frac{\hbar^2 \nabla^2}{2m} \phi -
  \vec{r} \frac{ke^2}{r}\phi = -\vec{r} \left( \frac{\hbar^2}{2m}\nabla^2 \phi
  + \frac{ke^2}{r}\phi \right)
\end{equation*}
Więc ostatecznie
\begin{equation*}
  \left[ \hat{H}_{at}, \vec{r} \right] \phi = \frac{-\hbar ^2 }{2m} \left(
    2 \nabla \phi  + \vec{r} \nabla^2 \phi \right) - \frac{ke^2}{r} (\vec{r}
    \phi) + \vec{r} \left( \frac{\hbar^2}{2m}\nabla^2 \phi
    + \frac{ke^2}{r}\phi \right)  = \frac{-i\hbar}{m} \hat{\vec{p}}
\end{equation*}

\section{Wykrzyknik 22}
	Hamiltonian układu
		$$
		\hat{H} = \hat{H}_{at} + \hat{H}'
		$$
	gdzie $\hat{H}'=\frac{e}{m}\vec{A}\circ\hat{\vec{p}}-$ hamiltonian opisujący oddziaływanie z polem elektromagnetycznym. Niech $\psi(\vec{r},t)$ spełnia równanie Schrodingera
		$$
		\hat{H}\psi(\vec{r},t)=i\hbar\partial_t \psi(\vec{r},t)
		$$
	funkcję falową rozkładamy w bazie wektorów własnych operatora $\hat{H}_{at}$, tzn. $\hat{H}_{at}\psi_j(\vec{r},t) = E_j\psi_j(\vec{r},t)$, wtedy
		$$
		\psi(\vec{r},t) = C_1(t)\psi_1(\vec{r},t) + C_2(t)\psi_2(\vec{r},t)
		$$
	Skoro ta funkcja falowa spełnia wcześniejsze równanie Schrodingera, to:
		$$
		\hat{H}\left[C_1\psi_1(\vec{r},t) + C_2\psi_2(\vec{r},t)\right] = i\hbar\partial_t\left[C_1\psi_1(\vec{r},t) + C_2\psi_2(\vec{r},t)\right]
		$$
	pamiętamy, że $\dot{\psi}_j = -i/\hbar E_j \psi_j$ oraz $\hat{H} = \hat{H}_{at} + \hat{H}'$, wtedy, po przekształceniach dochodzimy do postaci
		$$
		\hat{H}'C_1\psi_1 + \hat{H}'C_2\psi_2 = i\hbar\left(\dot{C}_1\psi_1+\dot{C}_2\psi_2\right)
		$$
	następnie, aby otrzymać jedno z równań, mnożymy obustronnie przez $\psi_1$, wówczas
		$$
		C_1\bra{\psi_1}\hat{H}'\ket{\psi_1}+C_2\bra{\psi_1}\hat{H}'\ket{\psi_2} = i\hbar\dot{C}_1
		$$
	wcześniej obliczaliśmy $\bra{\varphi_j}\hat{H}'\ket{\varphi_k} = W_{jk}\cos(\omega t)$, oczywiście $\psi_j = e^{-i\hbar E_j t}\varphi_j$
		$$
		i\hbar\dot{C}_1 = C_2 W_{12}e^{i\omega_{12}t}\cos(\omega t)
		$$
	oraz analogicznie postępujemy dla wektora $\psi_2$, co prowadzi do
		$$
		i\hbar\dot{C}_2 = C_1 W_{21}e^{-i\omega_{12}t}\cos(\omega t)
		$$
	oczywiście $\omega_0 = -\omega_{12}$ co od razu prowadzi do wyniku.
\subsection{Komentarze}
	Istotna uwaga jest taka, że $\hat{H}'$ zależy od $A_0^2$, ale skoro sama wartość $A_0$ jest mała to składnik z $A_0^2$ możemy pominąć.\\
	Wartość $\bra{\varphi_j}\hat{H}'\ket{\varphi_k} = W_{jk}\cos(\omega t)$ wyprowadziliśmy następująco: załóżmy, że mamy przybliżenie dipolowe, tzn. 
		$$
		\left.\frac{|\vec{k}\circ\vec{r}|}{\omega t}\right|_{t\sim\pi} \ll 1
		$$
	wówczas element macierzowy operatora $\hat{H}' = \frac{e}{m}\vec{A}\circ\hat{\vec{p}}$ przyjmie postać
		$$
		\bra{\varphi_j}\hat{H}'\ket{\varphi_k}=\frac{e}{m}\bra{\varphi_j}\cos(\omega t-\vec{k}\circ\vec{r})\vec{A}_0\circ\hat{\vec{p}}\ket{\varphi_k} = \frac{e}{m}\cos(\omega t)\vec{A}_0
		\bra{\varphi_j}\hat{\vec{p}}\ket{\varphi_k}
		$$
	tutaj rozwinęliśmy $\cos$ w szereg względem $kr$. Element macierzowy $\bra{\varphi_j}\hat{\vec{p}}\ket{\varphi_k}$ wyliczymy korzystając z z komutacji
		$$
		\left[\hat{H}_{at},\hat{\vec{r}}\right] = -i\hbar/m \hat{\vec{p}}
		$$
	wówczas
		$$
		\bra{\varphi_j}\hat{\vec{p}}\ket{\varphi_k} = \bra{\varphi_j}im/\hbar\left[\hat{H}_{at},\hat{\vec{r}}\right]\ket{\varphi_k}= i\frac{m}{\hbar}(E_j-E_k)\bra{\varphi_j}\vec{r}\ket{\varphi_k}
		$$
	wprowadzamy to wyrażenie do obliczonego elementu macierzowego, wprowadzając wektor elektrycznego momentu dipolowego $\vec{d}=e\vec{r}$
		$$
			\bra{\varphi_j}\hat{H}'\ket{\varphi_k} = \frac{e}{m}\cos(\omega t)\vec{A}_0i\frac{m}{\hbar}(E_j-E_k)\frac{1}{e}\vec{d}_{jk} = i\omega_{jk}\cos(\omega t)\vec{A}_0\circ\vec{d}_{jk} = W_{jk}\cos(\omega t)
		$$
	Przybliżenie dipolowe uzasadniamy faktem, że 
		$$
		\varphi(\vec{r}) \sim e^{-r/a n} \implies r \leq a n
		$$
	ponadto 
		$$
		k = \omega/c \sim \frac{E_0}{2\hbar c n^2}
		$$
	wtedy
		$$
		kr \leq \frac{\alpha}{2n} \ll 1
		$$

\section{Wykrzyknik 23}
Z równania na $\dot{C}_2$ wylicz $C_1$, zróżniczkuj (pamiętaj o pochodnej
iloczynu, od t zależy eksponenta oraz c) i wstaw do równania na $\dot{C}_1$. Po
przerzuceniu wszystkiego na jedną stronę otrzymujemy wynik. 
Rozwiązywanie rówania:
zakładamy
\begin{equation*}
  C_2 = e^{i\lambda t}
\end{equation*}
\begin{equation*}
  \dot{C_2} = i\lambda e^{i\lambda t}
\end{equation*}
\begin{equation*}
  \ddot{C_2} = -\lambda^2 e^{i\lambda t}
\end{equation*}
Równanie charakterystyczne
\begin{equation*}
  -\lambda^2 - \omega_{-} \lambda + \frac{\abs{W_{12}}^2}{4\hbar} = 0
\end{equation*}
Warunki początkowe z ćwiczeń
\begin{equation*}
	C_1(0) = 0
\end{equation*}
\begin{equation*}
	C_2(0) = 0
\end{equation*}
\begin{equation*}
	\dot{C_2}(0) = \frac{-i}{2\hbar}W_{12}^*
\end{equation*}
Stąd 
\begin{equation*}
  \lambda_{1,2} = \frac{-\omega_{-} \pm \omega_{R}}{2}
\end{equation*}
Więc otrzymujemy
\begin{equation*}
  C_2(t) = A exp\left[ i \frac{-\omega_{-} + \omega_{R}}{2} t\right] + B \left[
  i \frac{-\omega_{-} - \omega_{R}}{2} t\right] 
\end{equation*}
Z warunków początkowych wyliczamy A i B.
\begin{equation*}
  C_2(t) = \frac{-W_{12}^*}{\hbar \omega_r}e^{\frac{-i\omega_{-}}{2}t}
  \sin(\frac{\omega_r}{2}t)
\end{equation*}
Prawdopodobieństwo
\begin{equation*}
  P_{12} = \abs{C_2(t)}^2 = \frac{\abs{W_{12}}^2}{\hbar^2 \omega_r^2}
  \sin^2(\frac{\omega_r}{2}t)
\end{equation*}
Brakującą energię atom uzyskuje z pomiaru.

\section{Wykrzyknik 24}
	Przepiszmy układ z wykrzyknika 22.
	$$
	\begin{array}{l}
		\dot{C}_1 = -\frac{i}{\hbar}C_2 W_{12}e^{-i\omega_0 t}\cos(\omega t)\\\\
		\dot{C}_2 =  -\frac{i}{\hbar}C_1 W_{21}e^{i\omega_0t}\cos(\omega t)
	\end{array}
	$$
	dokonajmy następującego przekształcenia
	$$
	\cos(\omega t) e^{-i\omega_0t} = \frac{1}{2}\left( e^{i\omega t}+e^{-i\omega t} \right)e^{-i\omega_0t} = \frac{1}{2}\left(e^{i\omega_{-}t}+e^{-i\omega_{+}t}\right)
	$$
	analogicznie
	$$
	\cos(\omega t) e^{i\omega_0t} = \frac{1}{2}\left( e^{i\omega t}+e^{-i\omega t} \right)e^{i\omega_0t} = \frac{1}{2}\left(e^{i\omega_{+}t}+e^{-i\omega_{-}t}\right)
	$$
	wówczas układ przyjmuje postać
	$$
	\left\{\begin{gathered}
		\dot{C}_1 = -\frac{i}{2\hbar}C_2 W_{12} \left(e^{i\omega_{-}t}+e^{-i\omega_{+}t}\right)\\
		\dot{C}_2 =  -\frac{i}{2\hbar}C_1 W_{12}^* \left(e^{i\omega_{+}t}+e^{-i\omega_{-}t}\right)
	\end{gathered} \right.
	$$
	kiedy $\omega_0 \sim \omega$, wówczas $\omega_{+}\gg|\omega_{-}|$, stąd $T_- \gg T_+$.\\
	Będziemy średniować układ po czasie $t\gg T_+$. Zauważmy najpierw, że
	$$
	\frac{1}{T_+}\int\limits_t^{t+T_+}\dot{C}_id\tau = \frac{1}{T_+}\left[C_i(T_++t)-C_i(t)\right] = \overline{\dot{C}_i}
	$$
	z drugiej strony
	$$
	\frac{d}{dt}\frac{1}{T_+}\int\limits_t^{t+T_+}C_id\tau = \frac{1}{T_+}\frac{d}{dt}\left[\int\limits_0^{t+T_+}C_id\tau-\int\limits_0^t C_id\tau\right] = \dot{\overline{C_i}}
	$$
	zatem $\dot{\overline{C_i}} = \overline{\dot{C}_i}$.\\
	Przejdźmy do średniowania prawych stron układu równań(bez współczynników $W_{12}$- są niezależne od czasu), dla pierwszego z nich
	$$
	\begin{gathered}
		\frac{1}{T_+}\int\limits_t^{t+T_+}e^{i\omega_{-}\tau}C_2(\tau)d\tau + \frac{1}{T_+}\int\limits_t^{t+T_+}e^{-i\omega_{+}\tau}C_2(\tau)d\tau =
		\begin{vmatrix}
			u = C_2 & v'=e^{i\omega_{-}\tau}\\ \dot{u}=\dot{C}_2 & v = \frac{1}{i\omega_-}e^{i\omega_-\tau}
		\end{vmatrix} = \\
		= \left.\frac{C_2(\tau)}{i\omega_-}\right|_t^{t+T_+} - \frac{1}{i\omega_-}\int\limits_t^{t+T_+}e^{i\omega_-\tau}\dot{C}_2(\tau)d\tau - \left.\frac{C_2(\tau)}{i\omega_+}\right|_t^{t+T_+} + \frac{1}{i\omega_+}\int\limits_t^{t+T_+}e^{-i\omega_+\tau}\dot{C}_2(\tau)d\tau \\
		\bigg\vert \omega_+ \gg \omega_- \bigg\vert = \frac{1}{i\omega_-}\left[\left.C_2(\tau)\right|_t^{t+T_+} -\int\limits_t^{t+T_+}e^{i\omega_-\tau}\dot{C}_2(\tau)d\tau
		-\frac{\omega_-}{\omega_+}\left(\left.C_2(\tau)\right|_t^{t+T_+}-\int\limits_t^{t+T_+}e^{-i\omega_+\tau}\dot{C}_2(\tau)d\tau\right)\right]
	\end{gathered}
	$$
	zatem możemy zaniedbać drugi człon, czyli
	$$
	\frac{1}{T_+}\int\limits_t^{t+T_+}e^{i\omega_{-}\tau}C_2(\tau)d\tau + \frac{1}{T_+}\int\limits_t^{t+T_+}e^{-i\omega_{+}\tau}C_2(\tau)d\tau \approx
	\frac{1}{T_+}\int\limits_t^{t+T_+}e^{i\omega_{-}\tau}C_2(\tau)d\tau
	$$
	z drugiej strony w przedziale $[t,t+T_+]$ eksponenta jest wolnozmienna, zatem możemy ją wyłączyć przed znak całki
	$$
	\frac{1}{T_+}\int\limits_t^{t+T_+}e^{i\omega_{-}\tau}C_2(\tau)d\tau \approx e^{i\omega_{-}t}\frac{1}{T_+}\int\limits_t^{t+T_+}C_2(\tau)d\tau  = \overline{C_2}(t)e^{i\omega_{-}t}
	$$
	postępując analogicznie dla drugiego równania otrzymujemy
	$$
	\left\{\begin{gathered}
		\dot{\overline{C_1}} = -\frac{i}{2\hbar}W_{12} \overline{C_2}(t)e^{i\omega_{-}t}\\
		\dot{\overline{C_2}} =  -\frac{i}{2\hbar}W_{12}^* \overline{C_1}(t)e^{-i\omega_{-}t}
	\end{gathered} \right.
	$$

\section{Wykrzyknik 25}
Średniujemy po polaryzacjach
\begin{equation*}
  \langle p_{ab} \rangle_\lambda = \frac{1}{2} \sum_{\lambda=1}^{2} p_{ab}\left(
  \hat{e}_\lambda \right) = \pi \frac{\rho\left(w\right)}{\epsilon_0 \hbar^2}
  \frac{1}{2}\sum_{\lambda=1}^2 \abs{\hat{e}_\lambda \circ \vec{d}_{12}}^2
\end{equation*}
Niech (gdzie $\vec{W} = \vec{u} + i\vec{v}$ jest dowolnym wektorem w zespolonej przestrzeni trójwymiarowej)
\begin{equation*}
  S\left( \hat{q} \right) = \frac{1}{2}\sum_{\lambda=1}^2 \abs{\hat{e}_\lambda
  \circ \vec{W}}^2 = \frac{1}{2} \sum_{\lambda=1}^2 \abs{\vec{W}_\lambda}^2 =
  \frac{1}{2} \underbrace{\sum_{\lambda=1}^3 \abs{\vec{W_\lambda}}}_{\abs{\vec{W}}^2} - \frac{1}{2}\abs{\vec{W}
    \circ \hat{q}} = \frac{1}{2} \left( \abs{\vec{W}}^2 - \abs{\vec{W} \circ
	  \hat{q}} \right) 
\end{equation*}
Średniowanie po kierunkach
\begin{equation*}
  \langle S(\hat{q}) \rangle_{\hat{q}} = \frac{1}{4\pi} \int_{4\pi} d\Omega
  _{\hat{q}}
  S(\hat{q}) = \underbrace{\frac{1}{4\pi} \int_{4\pi}
  \frac{\abs{\vec{W}}}{2}d\Omega_{\hat{q}}}_{I_1}
  - \underbrace{\frac{1}{4\pi} \int_{4\pi} d\Omega_{\hat{q}}
  \frac{\abs{\vec{W} \circ \hat{q}}^2}{2}}_{I_2}
\end{equation*}
\begin{equation*}
  I_1 = \frac{\abs{\vec{W}}^2}{2}
\end{equation*}
\begin{equation*}
  I_2 = \frac{1}{8\pi} \int_{4\pi} d\Omega_{\hat{q}} \left( \abs{\vec{u} \circ
  \hat{q} }^2 + \abs{\vec{v} \circ \hat{q} }^2 \right) = \bigg \vert Oz
  \parallel \vec{u}, \vec{v} \bigg \vert = \frac{1}{8\pi} \left(  2\pi u^2 \int_{-1}^1
  dzz^2 + 2\pi v^2 \int_{-1}^1
dzz^2 \right) = \frac{1}{8\pi} \frac{4}{3}\pi \left( \vec{u}^2 + \vec{v}^2 \right)
\end{equation*}
wracając
\begin{equation*}
  \langle S(\hat{q}) \rangle_{\hat{q}} = I_1 + I_2 = \frac{\abs{\vec{W}^2}}{2} -
  \frac{1}{8\pi} \frac{4}{3}\pi \left( \vec{u}^2 + \vec{v}^2 \right) =
  \frac{1}{3} \abs{\vec{W}}^2
\end{equation*}
Więc
\begin{equation*}
  \langle p_{ab} \rangle_{\lambda, \hat{q}} = \frac{\pi}{3}
  \frac{\rho(\omega)}{\epsilon_0 \hbar^2} \abs{\vec{d}_{12}} 
\end{equation*}
korzystamy z równości
\begin{equation*}
  \rho(\omega) d\omega = \rho(\nu)d\nu
\end{equation*}
\begin{equation*}
  \rho(\omega) = \rho(\nu) \frac{d\nu}{d\omega} = \frac{\rho(\nu)}{2\pi}
\end{equation*}
oraz uzywamy stałej elektrostatycznej
\begin{equation*}
  \langle p_{ab} \rangle_{\lambda, \hat{q}} = \frac{2}{3} \pi
\frac{k\rho(\nu)}{\hbar^2} \abs{\vec{d}_{12}}
\end{equation*}


\section{Wykrzyknik 26}
\textbf{Radzę przeliczyć na własną rękę!}\\
	Zapis z $\alpha$
	$$
	\begin{gathered}
		A_{21} = \frac{4}{3}k_{el}\frac{d_{12}^2}{\hbar c^3}\omega_{21}^2 = \bigg\vert \alpha = \frac{k_{el} e^2}{\hbar c} \bigg\vert = \frac{4}{3} \frac{k_{el} e^2}{\hbar c}
				\frac{r_{12}^2}{c^2}\omega_{21}^2 =\\
				=\frac{4}{3} \alpha \frac{r_{12}^2}{c^2}\omega_{21}^2 = \frac{4}{3} \alpha \frac{r_{12}^2}{c^2} \frac{E_0}{\hbar}\left[-\frac{1}{8}+\frac{1}{2}\right] = 
				\bigg\vert E_0 = \alpha^2 (mc)^2\bigg\vert = \frac{1}{2}\alpha^3 \frac{r_{12}^2m^2}{\hbar}
	\end{gathered}
	$$
	Jednostka
	$$
	[A_{21}] = \frac{Nm^2}{C^2}\frac{C^2 m^2}{J\cdot s\cdot m^3/s^3}\frac{1}{s^2} = \frac{1}{s}
	$$
	Szacowana wartość
	$$
	A_{21} \sim 6,26 \cdot10^{8} \frac{1}{s}
	$$

\section{Wykrzyknik 27}
Tego też jako tako nie było, obliczenia prowizoryczne:
U nas stan $k$ to (2,1,1) a stan $j$ to (1,0,0).
Wyliczamy $d_{12}$
\begin{equation*}
  \vec{d}_{jk} = \bra{\phi_{100}} \vec{r}e \ket{\phi_{211}}
\end{equation*}
Używając wzorów z komentarzy
\begin{equation*}
  \vec{d}_{jk} = \sqrt{\frac{4\pi}{3}} \sum_{m=-1}^1 \hat{\epsilon}_m  
  \underbrace{\int_0^\infty dr r^2 R_{10}(r) R_{21}(r)}_{I} \int_{4\pi} d\Omega
  \underbrace{Y_{00}^*}_{\frac{1}{\sqrt{4\pi}}}Y_{11} 
  Y_{1m}^* 
\end{equation*}
\begin{equation*}
  \vec{d}_{jk} = \frac{e}{\sqrt{3}}\sum_{m=-1}^1 \hat{\epsilon}_m I
  \underbrace{\int_{4\pi} d\Omega Y_{11} Y_{1m}^*}_{\delta_{11} \delta_{1m}}
\end{equation*}
Więc
\begin{equation*}
  \abs{\vec{d}_{jk}}^2 = \frac{e^2}{3} I^2 \epsilon_1^2 = \frac{e^2}{3} I^2 
\end{equation*}
Z danych widzimy, że 
\begin{equation*}
  R_{01} = 2e^{-r}
\end{equation*}
\begin{equation*}
  R_{21} = \frac{\sqrt{6}}{12}e^{-\frac{r}{2}}
\end{equation*}
Więc
\begin{equation*}
  I = \frac{\sqrt{6}}{6} \int_0^\infty dr r^3 e^{-r} e^{-\frac{r}{2}} = \bigg
  \vert \int_0^\infty dr r^n e^{-ar} = \frac{n!}{a^{n+1}} \bigg \vert =
  \frac{16}{81} \sqrt{6}
\end{equation*}
\begin{equation*}
  \abs{\omega_{21}} = E_1 - E_2 = \bigg \vert E_n = -E \frac{Z^2}{2n^2} \bigg
  \vert = \abs{- \frac{1}{2} + \frac{1}{8}} = \frac{3}{8}
\end{equation*}

\subsection{Komentarze}
\begin{equation*}
  \vec{d}_{jk} = \bra{\phi_{n_j l_j m_j}} \vec{r}e \ket{ \phi_{n_k l_k m_k}}
\end{equation*}
\begin{equation*}
  \phi_{nlm} = Y_{lm} (\hat{r}) R_{nl} (r)
\end{equation*}
\begin{equation*}
  \vec{r} = r \sqrt{\frac{4\pi}{3}} \sum_{m=-1}^1 Y_{1m} \epsilon_{m}^{*}
\end{equation*}

\section{Wykrzyknik 28}
	Nie no... wystarczy umieć mnożyć macierze.
  Plus te dowody są na wikipedii.

\section{Wykrzyknik 29}
\begin{equation*}
  \vec{n} \circ \vec{S} = \frac{\hbar}{2} \left[ \begin{pmatrix} 0 & n_1 \\ n_1 &
      0 \end{pmatrix} + \begin{pmatrix} 0 & -in_2 \\ in_2 & 0 \end{pmatrix}
    + \begin{pmatrix} n_3 & 0 \\ 0 & -n_3 \end{pmatrix} \right] =
    \frac{\hbar}{2} \begin{pmatrix} n_3 & n_{-} \\ n{+} & -n_3 \end{pmatrix}
\end{equation*}
Gdzie wprowadzono oznaczenia
\begin{equation*}
  n_{\pm} = n_1 \pm in_2
\end{equation*}
Żeby istniało rozwiązanie wyznacznik musi być niezerowy
\begin{equation*}
  \frac{\hbar}{2} \begin{pmatrix} n_3 & n_{-} \\ n{+} & -n_3 \end{pmatrix}
  \hat{\phi} =
  \frac{\hbar}{2} \hat{\phi}
\end{equation*}
Obliczając wyznacznik i przyrównując do zera
\begin{equation*}
  -\left( n_3^2 -1\right) -n_1^2 - n_2^2 = -\vec{n}^2 + 1 = 0
\end{equation*}
Dostajemy układ równań
\begin{equation*}
  \left\{ \begin{gathered}
      (n_3 - 1)a + n_{-} b = 0\\
      n_{+} a - (n_3 +1) b = 0
  \end{gathered} \right\}
\end{equation*}
Przenosząc
\begin{equation*}
  \left\{\begin{gathered}
      n_{-} b = (1-n_3)a \qquad /\cdot a^*\\
      n_{+} a =  (n_3 +1) b \qquad /\cdot b^*
  \end{gathered}\right.
\end{equation*}
\begin{equation*}
  \left\{\begin{gathered}
      n_{-} b a^* = (1-n_3)\abs{a}^2  \\
      n_{+} a b^*=  (n_3 +1) \abs{b}^2 
  \end{gathered}\right.
\end{equation*}
Stąd
\begin{equation*}
  (1-n_3)\abs{a}^2 =(n_3 +1) \abs{b}^2  
\end{equation*}
\begin{equation*}
  \abs{a}^2 - \abs{a}^2n_3 = \abs{b}^2n_3 + \abs{b}^2
\end{equation*}
\begin{equation*}
  \abs{a}^2 - \abs{b}^2 = n_3 \underbrace{(\abs{a}^2 + \abs{b}^2)}_{=N^2} \quad
  \Rightarrow \quad n_3 = \frac{\abs{a}^2 - \abs{b}^2 }{N^2}
\end{equation*}
\begin{equation*}
  \left\{\begin{gathered}
      n_{-} \abs{b}^2 = (1-n_3)ab^*  \\
      n_{+} \abs{a}^2=  (n_3 +1)a^*b
  \end{gathered}\right.
\end{equation*}
Dodając stronami
\begin{equation*}
  n_{-} \abs{b}^2 + n_{+} \abs{a}^2= 2ab
\end{equation*}
\begin{equation*}
  n_{-} N^2 = 2ab \quad \Leftarrow \quad n_{-} \frac{2ab}{N^2}, \quad
  \frac{2a^*b}{N^2}
\end{equation*}
Więc ostatecznie
\begin{equation*}
  n_1 = \frac{ab^* + a^*b}{N^2}
\end{equation*}
\begin{equation*}
  n_2 = \frac{ab^* - a^*b}{N^2}
\end{equation*}

\section{Wykrzyknik 30}
Równanie Pauliego 
$$
\left\{\left[\frac{\left(\hat{\vec{P}}+e\vec{A}\right)^2}{2m}-e\Phi\right]\mathbb{I}-\hat{\vec{\mu}}_s\circ\vec{B} \right\}\hat{\varphi} = i\hbar\partial_t\hat{\varphi}
$$
Pomijając składniki rzędu $\frac{1}{m}$ 
$$
\left(-e\Phi - \hat{\vec{\mu}}_s\circ\vec{B}\right)\hat{\varphi} = i\hbar\partial_t\varphi
$$
ograniczając się jedynie do spinu
$$
\left(i\hbar\partial_t+\hat{\vec{\mu}}_s\circ\vec{B}\right)\hat{\varphi} = 0
$$
definujemy wektor $\vec{w}(t)$, który jest rozwiązaniem powyższego równania, jako
$$
\vec{w}(t) = \bra{\hat{\varphi}(t)}\hat{\vec{S}}\ket{\hat{\varphi}(t)}
$$
wówczas
$$
\dot{\vec{w}}(t) = \bra{\dot{\hat{\varphi}}}\hat{\vec{S}}\ket{\hat{\varphi}}+\bra{\hat{\varphi}}\hat{\vec{S}}\ket{\dot{\hat{\varphi}}}
$$
z naszego równania wyliczamy wartość pochodnej po czasie z funkcji $\varphi$
$$
=\bra{\frac{i}{\hbar}\hat{\vec{\mu}}_s\circ\vec{B}\hat{\varphi}}\hat{\vec{S}}\ket{\varphi} + \bra{\hat{\varphi}}\hat{\vec{S}}\ket{\frac{i}{\hbar}\hat{\vec{\mu}}_s\circ\vec{B}}
$$
teraz wprowadzamy jawną postać operatora $\hat{\vec{\mu}}_s= -g_s\mu_B\frac{\hat{\vec{S}}}{\hbar}$, oznaczamy przez $ia=\frac{1}{i\hbar}g_s\mu_B$ oraz zauważamy, że 
$$
\hat{\vec{S}} = \hat{S}_ie_i,\quad\quad \vec{S}\circ\vec{B} = S_jB_j
$$
wówczas
$$
=-ia\bra{\hat{S}_jB_j\hat{\varphi}}\hat{S}_i e_i\ket{\hat{\varphi}}+ia\bra{\hat{\varphi}}\hat{S}_ie_i\ket{\hat{S}_jB_j\hat{\varphi}} = ia\bra{\varphi}\left[\hat{\vec{S}},\hat{\vec{S}}\circ\vec{B}\right]\ket{\varphi} =ia e_i B_j \bra{\varphi}\left[S_i,S_j\right]\ket{\varphi}
$$
co, po skorzystaniu z reguły komutacji operatorów spinu, prowadzi do
$$
-\hbar a \epsilon_{ijk} B_j e_i w_k(t) =-\hbar a \vec{B}\times\vec{w}(t)
$$
Przechodzimy do rozwiązania równania, zauważmy, że
$$
\dot{\vec{w}}(t)= \frac{g_s\mu_B}{\hbar}\vec{B}\times\vec{w} \quad\quad \bigg| \circ \vec{w}
$$
po prawej stronie otrzymujemy $0$, zatem
$$
\frac{d}{dt}\vec{w}^2 = 0
$$
skoro długość wektora pozostaje stała w czasie, możemy wnioskować, że jest to ruch po okręgu, zatem
$$
\vec{w} = \vec{w}_{||} + \vec{w}_{\perp}
$$
otrzymujemy równanie
$$
\dot{\vec{w}} = C\vec{B}\times\vec{w}_\perp
$$
przyjmijmy ponadto, że $OZ||\vec{B}$
$$
\left\{
\begin{split}
\dot{w}_x &= -Cw_yB\\
\dot{w}_y &= Cw_xB
\end{split}
\right.
$$
układ równań możemy przekształcić do równania rzędu drugiego
$$
\ddot{w}_y+C^2B^2w_y = 0
$$
rozwiązanie takiego równania 
$$
A\cos(\lambda t) + B\sin(\lambda t)
$$
gdzie $\lambda = C B = \frac{g_s\mu_B B}{\hbar} = \omega_L$
Otrzymaliśmy zatem wzór na precesję Larmora.
\section{Wykrzyknik 31}
$$
\begin{gathered}
\left[\hat{L}_i,\hat{H}_D\right] = \left[\hat{L}_i,c\vec{\alpha}\circ\hat{\vec{p}}\right] + \left[\hat{L}_i,mc^2\beta \right] = c\left[\hat{L}_i,\vec{\alpha}\circ\hat{\vec{p}}\right] = c\left[\hat{L}_i, \alpha_j\hat{p}_j\right] = c\alpha_j \left[\hat{L}_i,\hat{p}_j\right] = \\
= c\alpha_j\left[\epsilon_{ivk}r_v\hat{p}_k,\hat{p}_j\right] = c\alpha_j\epsilon_{ivk}\left[r_v \hat{p}_k,\hat{p}_j\right] =  c\alpha_j\epsilon_{ivk}\left(r_v \hat{p}_k\hat{p}_j-r_v\hat{p}_j \hat{p}_k +i\hbar \delta_{vj}\hat{p}_k\right) = i\hbar c\alpha_j \epsilon_{ijk} \hat{p}_k = \\
= i\hbar c (\vec{\alpha}\times\hat{\vec{p}})_i
\end{gathered}
$$
$$
\begin{gathered}
\left[\hat{S}_i,\hat{H}_D\right] = \left[\hat{S}_i,c\vec{\alpha}\circ\hat{\vec{p}}\right] + \left[\hat{S}_i,mc^2\beta \right] = c\left[\hat{S}_i,\alpha_j\hat{p}_j\right] = c\frac{\hbar}{2}\left[
\begin{bmatrix} \sigma_i & \hat{0} \\ \hat{0} &\sigma_i \end{bmatrix}  ,\alpha_j\hat{p}_j\right]
 = c\frac{\hbar}{2}\hat{p}_j \left[\begin{bmatrix} \sigma_i & \hat{0} \\ \hat{0} &\sigma_i \end{bmatrix},\alpha_j\right] = \\  = c\frac{\hbar}{2}\hat{p}_j \left( 
\begin{bmatrix}
\hat{0} & \sigma_i\sigma_j\\
\sigma_i\sigma_j & \hat{0}
\end{bmatrix} - \begin{bmatrix}
\hat{0} & \sigma_j\sigma_i\\
\sigma_j\sigma_i & \hat{0}
\end{bmatrix}
 \right) =c\frac{\hbar}{2}\hat{p}_j \bigg(\begin{bmatrix}
 \hat{0} & i\epsilon_{ijk}\sigma_k + \delta_{ij}\\
 i\epsilon_{ijk}\sigma_k + \delta_{ij} & \hat{0}
 \end{bmatrix} - \\ - \begin{bmatrix}
 \hat{0} & i\epsilon_{jik}\sigma_k + \delta_{ji}\\
 i\epsilon_{jik}\sigma_k + \delta_{ij} & \hat{0}
 \end{bmatrix} \bigg) =c\frac{\hbar}{2}\hat{p}_j 2i\alpha_k \epsilon_{ijk} = -i\hbar c(\vec{\alpha}\times\hat{\vec{p}})_i
\end{gathered}
$$
co już w bardzo prosty sposób prowadzi do szukanych komutatorów.

\section{Wykrzyknik 32}
 	Obliczmy najpierw $\hat{S}\circ\hat{L}$ 
 		$$
 		\hat{J}= \hat{L}+\hat{S} \implies \hat{J}^2 = \hat{L}^2+\hat{S}^2 + 2\hat{S}\circ\hat{L}
 		$$
 	zatem
 		$$
 		\hat{S}\circ\hat{L}=\frac{1}{2}\left[\hat{J}^2-\hat{L}^2-\hat{S}^2 \right]
 		$$
 	wprowadzamy to do naszego hamiltonianu na oddziaływanie spin-orbita 
 		$$
 		\hat{H}_{SO} = \frac{-ke^2}{4m^2c^2}\frac{1}{r^3}\left[\hat{J}^2-\hat{L}^2-\hat{S}^2 \right]
 		$$
 	wtedy element macierzowy
 		$$
 		\begin{gathered}
 			\bra{nlsjm_j}\hat{H}_{SO}\ket{nlsjm_j} = const\cdot\bra{nlsjm_j}\frac{1}{r^3}\left(\hat{J}^2-\hat{L}^2-\hat{S}^2\right)\ket{nlsjm_j} = \\ =const\cdot\bra{nlsjm_j}\frac{1}{r^3}\left[j(j+1) - l(l+1)-s(s+1)\right]\ket{nlsjm_j} = \bigg\vert l = 0\bigg\vert = \\ = const\cdot\bra{nlsjm_j}\frac{1}{r^3}\left[j(j+1) -s(s+1)\right]\ket{nlsjm_j} = \bigg\vert j = |l\pm s| \bigg\vert = 0
 		\end{gathered}
 		$$
Jawna postać spinorbitali sferycznych Pauliego
$$
\hat{\varphi}_{nlsjm_j}^{0} = R_{nl}\begin{bmatrix}
\hat{\Omega}_{lm_lsjm_j} \\ 
\mathbb{O} \end{bmatrix} =  R_{nl}\sum\limits_{m_l,m_s}C^{j,m_j}_{m_l,m_s} Y_{l,m_l}\begin{bmatrix}\hat{\varphi}_{s,m_s} \\ \mathbb{O} \end{bmatrix}
$$

\section{Wykrzyknik 33}
	 Zsumujmy poprawki
	 $$
	 \lambda E^{(1)} = \Delta E_m + \Delta E_{SO} + \Delta E_{DAR}
	 $$
	 Rozważmy przypadki
		 \begin{enumerate}
		 	\item $l = 0$, wówczas
		 		\begin{itemize}
		 			\item[$\bullet$] $\Delta E_m = \frac{(Z\alpha)^2}{n^2}|E_n|\left(\frac{3}{4}  -\frac{n}{\frac{1}{2}} \right) =\frac{(Z\alpha)^2}{n^2}|E_n|\left(\frac{3}{4} -2n \right) $
		 			\item[$\bullet$] $\Delta E_{SO} = 0$
		 			\item[$\bullet$] $\Delta E_{DAR} = \frac{(Z\alpha)^2}{n}|E_n| $
		 		\end{itemize}
	 			co prowadzi do 
	 			$$
	 			\begin{gathered}
		 			\lambda E^{(1)} = \Delta E_m + \Delta E_{SO} + \Delta E_{DAR} = \frac{(Z\alpha)^2}{n^2}|E_n|\left(\frac{3}{4} -2n \right) + \frac{(Z\alpha)^2}{n}|E_n| = \\
		 			\frac{(Z\alpha)^2}{n^2}|E_n| \left(\frac{3}{4} -2n + n \right) = \frac{(Z\alpha)^2}{n^2}|E_n| \left(\frac{3}{4} -n  \right) = -\frac{(Z\alpha)^2}{n^2}|E_n| \left(n-\frac{3}{4}   \right)
	 			\end{gathered}
	 			$$
	 		\item $l\neq 0\quad j = l - \frac{1}{2}$, wtedy 
	 				\begin{itemize}
	 					\item[$\bullet$] $\Delta E_m = \frac{(Z\alpha)^2}{n^2}|E_n|\left(\frac{3}{4}  -\frac{n}{l+\frac{1}{2}} \right) $
	 					\item[$\bullet$] $\Delta E_{SO} = -\frac{(Z\alpha)^2}{n}|E_n|\frac{1}{2l+1}\frac{1}{l} $
	 					\item[$\bullet$] $\Delta E_{DAR} = 0  $
	 				\end{itemize}
 				co daje
 				$$
 				\begin{gathered}
 					\lambda E^{(1)} = \frac{(Z\alpha)^2}{n^2}|E_n|\left(\frac{3}{4}  -\frac{n}{l+\frac{1}{2}} \right) - \frac{(Z\alpha)^2}{n}|E_n|\frac{1}{2l+1}\frac{1}{l} =
 					-\frac{(Z\alpha)^2}{n^2}|E_n|\left[  \frac{n}{l+\frac{1}{2}}  - \frac{3}{4} + \frac{n}{2l(l+\frac{1}{2})}   \right] =\\
 					=-\frac{(Z\alpha)^2}{n^2}|E_n|\left[ \frac{(2l+1)n}{2l(l+\frac{1}{2})} - \frac{3}{4}\right] = -\frac{(Z\alpha)^2}{n^2}|E_n|\left[ \frac{n}{l} - \frac{3}{4}\right] 
 					= -\frac{(Z\alpha)^2}{n^2}|E_n|\left[ \frac{n}{j+\frac{1}{2}} - \frac{3}{4}\right] 
 				\end{gathered}
 				$$
 			\item $ l\neq 0\quad j = l + \frac{1}{2}$, wówczas
 				\begin{itemize}
 					\item[$\bullet$]  $\Delta E_m = \frac{(Z\alpha)^2}{n^2}|E_n|\left(\frac{3}{4}  -\frac{n}{l+\frac{1}{2}}\right) $
 					\item[$\bullet$]  $\Delta E_{SO} = \frac{(Z\alpha)^2}{n}|E_n|\frac{1}{2l+1} \frac{1}{l+1}$
 					\item[$\bullet$]  $\Delta E_{DAR} = 0 $
 				\end{itemize}
 				więc
 				$$
 				\begin{gathered}
 					\lambda E^{(1)} =  \frac{(Z\alpha)^2}{n^2}|E_n|\left(\frac{3}{4}  -\frac{n}{l+\frac{1}{2}}\right) +  \frac{(Z\alpha)^2}{n}|E_n|\frac{1}{2l+1} \frac{1}{l+1} =\\ -\frac{(Z\alpha)^2}{n^2}|E_n| \left[  \frac{n}{l+\frac{1}{2}} - \frac{3}{4} - \frac{n}{(2l+1)(l+1)}  \right] =-\frac{(Z\alpha)^2}{n^2}|E_n|
 					  \left[ \frac{2n(l+1) - n}{(2l+1)(l+1)} - \frac{3}{4} \right] = \\
 					  =-\frac{(Z\alpha)^2}{n^2}|E_n|\left[ \frac{n(2l+1)}{(2l+1)(l+1)} - \frac{3}{4} \right] = -\frac{(Z\alpha)^2}{n^2}|E_n|\left[ \frac{n}{j+\frac{1}{2}} - \frac{3}{4} \right] 
 				\end{gathered}
 				$$
		 \end{enumerate}

\section{Wykrzyknik 34}
MISSING :V

\section{Wykrzyknik 35}
	Przejście graniczne $\alpha \to 0$ jest równoważne przejściu $\gamma \to 1$. Rozwiązanie 
	$$
	\hat{\psi}_{1,1/2,1/2}(\vec{r}) = f(r)
		\begin{pmatrix}
			1 \\ 0 \\ i\frac{1-\gamma}{Z\alpha}\cos\theta \\ i\frac{1-\gamma}{Z\alpha}\sin\theta e^{i\varphi}
		\end{pmatrix}
	$$
	Obliczmy jego granicę
	$$
	\lim\limits_{\gamma\to 1} f(r)
	\begin{pmatrix}
		1 \\ 0 \\ i\frac{1-\gamma}{\sqrt{1-\gamma^2}}\cos\theta \\ i\frac{1-\gamma}{\sqrt{1-\gamma^2}}\sin\theta e^{i\varphi}
	\end{pmatrix}
	$$
	Dla funkcji $f(r)$
	$$
	\lim\limits_{\gamma\to 1}f(r) =\frac{1}{\sqrt{4\pi}}\left(\frac{2Z}{a_0}\right)^{3/2} e^{-Zr/a_0}
	$$
	osobno policzymy granicę
	$$
	\lim\limits_{\gamma\to 1}\frac{1-\gamma}{\sqrt{1-\gamma^2}} = \lim\limits_{\gamma\to 1}\frac{1-\gamma}{\sqrt{1-\gamma}\sqrt{1+\gamma}} = 0
	$$
	zatem
	$$
	\lim\limits_{\gamma\to 1}\hat{\psi}_{1,1/2,1/2}(\vec{r}) = \frac{1}{\sqrt{4\pi}}\left(\frac{2Z}{a_0}\right)^{3/2} e^{-Zr/a_0}
	\begin{pmatrix}
		1 \\ 0 \\ 0 \\ 0
	\end{pmatrix} = R_{10}(r) Y_{00}\begin{pmatrix}
	\hat{\varphi}_+ \\ 0 \\ 0
	\end{pmatrix}
	$$
	analogicznie dla $-1/2$.
\section{Wykrzyknik 36}
Zaczynamy od Hamiltonianu
$$
\hat{H}=\hat{H}_{at} + \hat{H}_{rel} + \hat{H}_B
$$
przy czym zakładając, że $OZ||\vec{B}$, możemy zapisać
$$
\hat{H}_B = \frac{\mu_B}{\hbar}B\left(\hat{L}_z + 2\hat{S}_z\right)
$$
dla anomalnego efektu Zeemanna $<H_{rel}> \gg <H_B>$, przy słabym $\vec{B}$. Wtedy 
obydwie poprawki traktujemy jako zaburzenia. Funkcje niezaburzone są rozwiązaniami równania Schrodingera, dostosawnymi do sumy poprawek,czyli
$$
\hat{H}_{at}\hat{\varphi}^{(0)} = E^{(0)}\hat{\varphi}^{(0)}
$$
macierz poprawek relatywistycznych jest diagonalna w bazie spinorbitali sferycznych Pauliego, zatem trzeba pokazać, że
$$
\bra{nl'j'm_j'}\hat{H}_B\ket{nljm_j}
$$
jest diagonalna, zatem trzeba obliczyć element macierzowy $\hat{H}_B = \frac{\mu_B B}{\hbar}\left(\hat{J}_z+\hat{S}_z \right)$. Obliczymy jednak jedynie element na diagonalii, gdyż to jest nam potrzebne do poprawki
$$
\begin{gathered}
\bra{nljm_j}\frac{\mu_B B}{\hbar}\left(\hat{J}_z+\hat{S}_z \right)\ket{nljm_j} = \mu_Bm_j B + \frac{\mu_B B}{\hbar}\underbrace{\bra{nlsjm_j}\hat{S}_z\ket{nlsjm_j}}_{x}
\end{gathered}
$$
tutaj korzystamy z podanej tożsamości wstawiając za operator $\hat{V}$ operator 
$\hat{\vec{S}}$. Potrzebujemy obliczyć
$$
\bra{nlsjm_j}\left[\hat{\vec{J}}^2,\left(\hat{\vec{J}}^2,\hat{S}_3\right)\right]\ket{nlsjm_j} = 
\bra{nlsjm_j}\left[ 2\hbar^2\left\{\hat{\vec{J}}^2,\hat{S}_3  \right\}-4\hbar^2\left(\hat{\vec{S}}\circ\hat{\vec{J}}\right)\hat{J}_3\right] \ket{nlsjm_j} 
$$
tożsamość, która przydaje się podczas obliczeń $\hat{\vec{L}}^2 = \hat{\vec{J}}^2+\hat{\vec{S}}^2- 2 \hat{\vec{J}}\circ\hat{\vec{S}}$.\\
\textbf{obliczenie lewej i prawej strony nie jest trudne, jest żmudne, zatem tego nie piszę}.
Po przekształceniach otrzymuje się wynik
$$
 x =\frac{m_j \hbar}{2j(j+1)}\left[j(j+1)+\frac{3}{4}-l(l+1)\right]
$$
co ostatecznie daje nam poprawkę na energię związaną z polem magnetycznym
$$
E = \mu_B m_j B \left[ 1+\frac{j(j+1) +s(s+1)-l(l+1)}{2j(j+1)}  \right]
$$
\section{Wykrzyknik 37}
Ogólnie czynnik Landego
$$
g_{jl} = 1+\frac{j(j+1) +s(s+1)-l(l+1)}{2j(j+1)}
$$
wiemy natomiast, że $s=1/2$, wówczas
$$
g_{jl} = 1+\frac{j(j+1) +\frac{3}{4}-l(l+1)}{2j(j+1)}
$$
Zanim przejdziemy do obliczeń to warto zauważyć, że 
$$
\frac{3}{4}-l(l+1) = -(l+3/2)(l-1/2)
$$
\begin{enumerate}
	\item jeżeli założymy tę samą liczbę $l$, przy $s=1/2$ to zgodnie ze zmianą liczby kwantowej $j$
	$$
	j = |l-1/2|,\quad j=l+1/2
	$$
	\begin{enumerate}
		\item $ j = |l-1/2|$, wówczas, przyjmijmy, że $|l-1/2|=l-1/2$, wtedy
		$$
		\begin{gathered}
			g_{lj} = 1 + \frac{(l-\frac{1}{2})(l+\frac{1}{2})-(l+3/2)(l-1/2)}{2(l-\frac{1}{2})(l+\frac{1}{2})} = 1+\frac{l+1/2-l-3/2}{2l+1} = \frac{2l}{2l+1}
		\end{gathered}
		$$
		\item $j=l+1/2$, postępując analogicznie upraszczamy ułamek
		$$
		g_{lj} = 1 + \frac{(l+\frac{1}{2})(l+\frac{3}{2})-(l+3/2)(l-1/2)}{2(l+\frac{1}{2})(l+\frac{3}{2})} = 1+\frac{1}{2l+1} = \frac{2l+2}{2l+1}
		$$
	\end{enumerate}
\item Próbowałem narysować kredką na monitorze, ale się nie udało.
\item $\Delta E = \mu_B B m_j g_{lj}$. W szczególności dla stanu $np_{1/2}$ otrzymujemy $g = 2/3$ oraz $\Delta E = \pm 1/3 \mu_B B$. Dla stanu $np_{3/2}$ dostajemy $g = 4/3$ oraz $\Delta E = \pm 2\mu_B B$ lub $\Delta E = \pm 2/3 \mu_B B$.
\end{enumerate}
\section{Wykrzyknik 38}
Zamiast elementu macierzowego obliczymy problem własny i wykażemy w ten sposób, że wartość własna to suma energii, zatem
$$
\hat{H}_0 \ket{\hat{\phi}_\gamma} = \sum_i^Z \hat{h}_i \ket{\hat{\phi}_\gamma} = \sum_i^Z \hat{h}_i\sum_p(-1)^{Inv(p)}\frac{1}{\sqrt{Z!}}\hat{P}\left[ \varphi_1(\vec{r}_1)\otimes\varphi_2(\vec{r}_2)\otimes\cdots\otimes\varphi_Z(\vec{r}_Z)  \right]
$$
zamieniamy trochę kolejność wyrazów
$$
\sum_i^Z\frac{1}{\sqrt{Z!}}\sum_p (-1)^{Inv(p)}\hat{h}_i\hat{P}\left[ \varphi_1(\vec{r}_1)\otimes\cdots\otimes\varphi_Z(\vec{r}_Z) \right]
$$
należy zauważyć, że operator $\hat{h}_i$ oraz $\hat{P}$ komutują ze sobą (ze względu na to, że hamiltonian jest sumą poszczególnych operatorów), wówczas
$$
\sum_i^Z\frac{1}{\sqrt{Z!}}\sum_p (-1)^{Inv(p)}\hat{P}\hat{h}_i\left[ \varphi_1(\vec{r}_1)\otimes\cdots\otimes\varphi_Z(\vec{r}_Z) \right]
$$
Zgodnie z własnościami iloczynu tensorowego $i-$ty hamiltonian 'odnajduje' jedynie $i-$ tą funkcję $\varphi_i$, zatem
$$
\sum_i^Z\frac{1}{\sqrt{Z!}}\sum_p (-1)^{Inv(p)}\hat{P}E_i\left[ \varphi_1(\vec{r}_1)\otimes\cdots\otimes\varphi_Z(\vec{r}_Z) \right]
$$
składając wszystko do całości
$$
\hat{H}_0 \ket{\hat{\phi}_\gamma} = 
\sum_i^ZE_i\frac{1}{\sqrt{Z!}}\sum_p (-1)^{Inv(p)}\hat{P}\left[ \varphi_1(\vec{r}_1)\otimes\cdots\otimes\varphi_Z(\vec{r}_Z) \right] = \sum_i^ZE_i\ket{\hat{\phi}_\gamma}
$$
co kończy dowód.
\section{Wykrzyknik 39}
\begin{itemize}
	\item $(np,n'p)$, no to składamy
	$$
	\begin{gathered}
	L = 1\oplus1 = 0,1,2 \\
	S = 1/2 \oplus 1/2 = 0,1
	\end{gathered}
	$$
	teraz tworzymy wszystkie możliwe kombinacje
	$$
	{}^1S,\,\,{}^3S,\,\,{}^1P,\,\,{}^3P,\,\,{}^1D,\,\,{}^3D
	$$
	\item $(np,n'd)$, again
	$$
	\begin{gathered}
	L = 1\oplus2 = 1,2,3 \\
	S = 1/2 \oplus 1/2 = 0,1
	\end{gathered}
	$$
	teraz tworzymy wszystkie możliwe kombinacje
	$$
	{}^1P,\,\,{}^3P,\,\,{}^1D,\,\,{}^3D,\,\,{}^1F,\,\,{}^3F
	$$
	\item $np^2$, robimy tabelę zawierającą nieujemne wartości rzutów $M_L, M_S$
	\begin{table}[H]
		\centering
		\begin{tabular}{|c|c|c|}\hline
			$M_L/M_S$ & 1 & 0 \\ \hline 
			2 & - & $1^+ 1^-$\\\hline 
			1 & $1^+ 0^+$ & $1^+0^-;1^-0^+$\\\hline 
			0 & $1^+-1^+$ & $1^+-1^-;1^--1^+;0^+0^-$ \\\hline
		\end{tabular}
	\end{table}
Dlaczego jedynie wartości nieujemne? Bo tyle wystarczy do znalezienia termów. Schemat postępowania
\begin{enumerate}
	\item zaczynamy od samej góry, widzimy, że wartość $M_L=2$ może odpowiadać momentom orbitalnym skierowanym w górę, zatem mamy do czynienia z termem $D$, jednakże, zgodnie z tabelą, widzimy, że nie ma możliwości ułożenia dwóch spinów w tym samym kierunku (zakaz Pauliego), zatem bierzemy drugą opcję, gdzie $M_S=0$, co odpowiada wypadkowemu spinowi równemu $S=0$, zatem $2S+1=1$, czyli pierwszy term jaki znaleźliśmy to ${}^1D$, skreślamy tę kombinację liczb.
	\item jak już mamy pierwszy term, to przystępujemy do skreślania. Jak wiemy, $L=2$ odpowiadają rzuty od $-2$ do $2$, zatem z każdego wiersza i kolumny skreślamy TYLKO PO JEDNEJ(ale dowolnej!) kombinacji liczb, które mogą również odpowiadać temu termowi, przykładowo, mając $L=2$ oraz $S=0$, możemy otrzymać kombinacje rzutów np. $M_L=1,M_S=0$, ale również
	 $M_L=0,M_S=0$. Widzimy, że w tabeli dla $M_L=1,M_S=0$ mamy kombinację liczb $1^+0^-;1^-0^+$, ale jak mówiłem skreślamy jedno, dowolne, tak samo dla $M_L=0,M_S=0$ mamy, aż 3 kombinacje, $1^+-1^-;1^--1^+;0^+0^-$, ale skreślamy jedną.\\
	 Czyli tabelka po pierwszym skreślaniu dla $\textcolor{red}{{}^1D}$ wygląda następująco
	 \begin{table}[H]
	 	\centering
	 	\begin{tabular}{|c|c|c|}\hline
	 		$M_L/M_S$ & 1 & 0 \\ \hline 
	 		2 & - & $\textcolor{red}{1^+ 1^-}$\\\hline 
	 		1 & $1^+ 0^+$ & $\textcolor{red}{1^+0^-};1^-0^+$\\\hline 
	 		0 & $1^+-1^+$ & $\textcolor{red}{1^+-1^-};1^--1^+;0^+0^-$ \\\hline
	 	\end{tabular}
	 \end{table}
	 \item Zatem skończyliśmy część dla termu ${}^1D$, teraz idąc od lewej od góry, znajdujemy pierwszą nieskreśloną kombinację, dla niej odczytujemy wartości $M_L = 1, M_S = 1$, co odpowiada termowi ${}^3P$. Skreślamy
	 \item dla ${}^3P$ możemy mieć $M_L=1,M_S=1$ ale również $M_L=1,M_S=0$, ponadto $M_L=0,M_S=1$ lub $M_L=0,M_S=0$. Tak jak wcześniej, skreślamy po jednym! Czyli dla $\textcolor{green}{{}^3P}$
	 \begin{table}[H]
	 	\centering
	 	\begin{tabular}{|c|c|c|}\hline
	 		$M_L/M_S$ & 1 & 0 \\ \hline 
	 		2 & - & $\textcolor{red}{1^+ 1^-}$\\\hline 
	 		1 & $\textcolor{green}{1^+ 0^+}$ & $\textcolor{red}{1^+0^-};\textcolor{green}{1^-0^+}$\\\hline 
	 		0 & $\textcolor{green}{1^+-1^+}$ & $\textcolor{red}{1^+-1^-};\textcolor{green}{1^--1^+};0^+0^-$ \\\hline
	 	\end{tabular}
	 \end{table}
 	\item no i odczytujemy ostatnią możliwość, czyli term ${}^1S$.
\end{enumerate}
	\item $nd^2$, czyli zabawa w to samo, tylko z większą ilością rzutów \textbf{stanów z zakazu Pauliego nie zapisuję!}
		\begin{table}[H]
		\centering
		\begin{tabular}{|c|c|c|}\hline
			$M_L/M_S$ & 1 & 0 \\ \hline 
			4 & - & $2^+ 2^-$\\\hline 
			3 & $2^+ 1^+$ & $2^+1^-;2^-1^+$\\\hline 
			2 & $2^+0^+$ & $2^+0^-;2^-0^+;1^+1^-$ \\\hline
			1 & $ 2^+ -1^+;1^+0^+$&$2^+-1^-;2^--1^+;1^+0^-;1^-0^+ $ \\\hline
			0 &$2^+-2^+;1^+-1^+$ & $2^+-2^-;2^--2^+;1^+-1^-;1^--1^+;0^+0^-$\\\hline
		\end{tabular}
	\end{table}
	otrzymane termy to
	$$
	\textcolor{red}{{}^1G},\,\,\textcolor{green}{{}^3F},\,\,\textcolor{blue}{{}^1D},\,\,\textcolor{gray}{{}^3P},\,\, {}^1 S
	$$
		\begin{table}[H]
		\centering
		\begin{tabular}{|c|c|c|}\hline
			$M_L/M_S$ & 1 & 0 \\ \hline 
			4 & - & $	\textcolor{red}{2^+ 2^-}$\\\hline 
			3 & $\textcolor{green}{2^+ 1^+}$ & $	\textcolor{red}{2^+1^-};\textcolor{green}{2^-1^+}$\\\hline 
			2 & $\textcolor{green}{2^+0^+}$ & $	\textcolor{red}{2^+0^-};\textcolor{green}{2^-0^+};\textcolor{blue}{1^+1^-}$ \\\hline
			1 & $ \textcolor{green}{2^+ -1^+};\textcolor{gray}{1^+0^+}$&$	\textcolor{red}{2^+-1^-};\textcolor{green}{2^--1^+};\textcolor{blue}{1^+0^-};\textcolor{gray}{1^-0^+ }$ \\\hline
			0 &$\textcolor{green}{2^+-2^+};\textcolor{gray}{1^+-1^+}$ & $	\textcolor{red}{2^+-2^-};\textcolor{green}{2^--2^+};\textcolor{blue}{1^+-1^-};\textcolor{gray}{1^--1^+};0^+0^-$\\\hline
		\end{tabular}
	\end{table}
\end{itemize}
\section{Wykrzyknik 40}
Dla termu ${}^3P$ mamy następujące wartości liczby $J$
$$
J =1\oplus1 = 0,1,2
$$
bogatsi o tę wiedzę widzimy, że dostaniemy 3 wartości poprawki
$$
\begin{gathered}
\Delta_{J=0}E_{SO} = \frac{\xi}{2}\left[0*1-1*(1+1)-1*(1+1)\right] = -2\xi \\
\Delta_{J=1}E_{S0} = \frac{\xi}{2}\left[1*(1+1)-1*(1+1)-1*(1+1) \right] = -\xi\\
\Delta_{J=2}E_{S0} = \frac{\xi}{2}\left[2*(2+1)-1*(1+1)-1*(1+1)  \right] =0
\end{gathered}
$$
no i widzimy, że dla $\xi<0$ najwyżej położony będzie term ${}^3P_0$, najniżej $ {}^3P_2$
\section{Wykrzyknik 41}
\begin{enumerate}
	\item Reguła Landego
	$$
	\begin{gathered}
	\Delta E_{SO}(J)-\Delta E_{SO}(J-1) = \frac{\xi}{2} \left\{ \left[ J*(J+1)-L(L+1)-S(S+1)\right]\right.-\\
	\left.\left[(J-1)*J-L(L+1)-S(S+1) \right]\right\} = \frac{\xi}{2}\left[J^2+J-J^2+J\right] = \xi J
	\end{gathered}
	$$
	\item Multiplety
	\begin{enumerate}
		\item Prosty $\xi>0 \leftrightarrow $podpowłoki zapełnione mniej, niż w połowie
		
		\item Odwrócony $\xi<0 \leftrightarrow$ podpowłoki zapełnione więcej, niż w połowie
	
	\end{enumerate}
\end{enumerate}
\end{document}
